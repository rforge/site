\title{Collaborative Software Development using R-Forge}
\author{Stefan Theu\ss{}l and Achim Zeileis}

\maketitle

%% Story of the article:
%% (1) Open source software (OSS) development - history, importance
%%     - Apache, Linux famous examples
%%     - Scripting languages, PHP, Perl and of course R
%%     - software repositories: CPAN, CTAN -> CRAN 
%% (2) collaboration and project management
%%     - collaboration sites: SourceForge.net
%% (3) why is it important to have a central repository
%%     - The Cathedral and the Bazaar
%%     - large community -> better
%% (4) - R-Forge
%%     - a set of tools
%% (5) How to forge packages on R-Forge (A dy in the life of an R
%%     package developer.
%%     - registering as a user
%%     - registering a project (automatically creates SVN, a
%%       mailinglist -> managable through web interface -> click to send
%%       commit statements to this list).
%%     - the svn base tree (picture of an axample tree with more
%%       packages)
%%     - R specific features
%%       Package building/checking facilities
%%     - quality management through R check and tracker
%%     - other stuff: forums/


\section*{Introduction}


%% from useR abstract 
%% <useR>

%% Why should software developers use source code management tools?

A key factor in open source software (OSS) development is the rapid creation
of solutions within an open, collaborative environment. The open
source model had its major breakthrough with the increasing
usage of the internet. Online communities successfully combined
not only their programming effort but also their knowledge, work
and even their social life. OSS development is generally
claimed to produce more bug free 
code and to be available faster than closed source code because of the
``release early, release often'' paradigm introduced by Linus
Torvalds.

All of the above led to an increasing demand for centralized resources
enabling developers to manage their projects and their source code as
well as to have standardized channels for communication. The most
famous of such platforms---the world's largest open source
software development web site---is SourceForge.net.

For a decade, the \R Development Core Team as well as many
\R package developers have been using
development tools like Subversion (SVN) or Concurrent Versions System
(CVS) for managing their source code. 
A central repository is hosted by ETH Z\"urich mainly for
managing the development of the base \R system. The
\R-project is now ready to provide similar infrastructure
for the entire \R community.

\R-Forge~(\url{http://R-Forge.R-project.org}) is a set of tools based
on the open source software GForge---a fork of
the open source version of SourceForge.net.
It aims to provide a platform for collaborative development of
\R packages, \R 
related software or other projects which are somehow related to \R.
%%our all beloved language \R
It offers source code management facilities through SVN and
a wide variety of web-based services.

%% brief overview of features possibly most interesting for readers
Furthermore, packages hosted on \R-Forge are built daily
for  various operating systems, i.e., Linux, MacOSX and Windows. These
package builds are downloadable from the 
project's website on \R-Forge as well as installable 
directly in \R via \code{install.packages()}.

%% Article Outline
The present article is organized as follows. First, we present the core
features R-Forge is offering to the \R community. Second we
show how package developers can get started with 
\R-Forge. In particular we show how people can register a project, 
use \R-Forge's source code management facilities, provide their
packages with \R-Forge, host a project specific website, and
finally submit a package to CRAN (\url{CRAN.R-project.org}). Then ...
Eventually we summarize ... and give a brief outlook and what we are
planning to offer in the future.


%% What is R-Forge offering to you?
%% Alternative title: Core Features of R-Forge
\section{R-Forge}
%% what is R-Forge?
R-Forge (\url{http://R-Forge.R-project.org}) is a set of tools based
on the open source software GForge (\cite{manual:gforge})%% or: a system
 for collaborative development of \R Packages, \R
related software or other projects which are somehow related to our
all beloved language \R. It provides source code management facilities
through Subversion (\cite{subversion07}) and various web-based features. 


%% Why should software developers use source code management tools?

R-Forge hosted nearly 200 Projects and had around 450 registered users
in its development stage. This and the steadily growing list of
feature requests show that there is a high demand for centralized source code
management tools and for releasing prototype code frequently among the
\R community.

%% What is GForge !!!! CITATION !!!!
GForge is a fork of the 2.61 SourceForge code. Since the codebase of
SourceForge has not been released over a certain time the GForge
project was formed. One of the authors of the original SourceForge
code, Tim Perdue, now maintains the GForge project.


%% FIXME: How to formulate better
%% Version control - probably the most important feature
In science people want to work together and exchange knowledge and
this is also true for developing software. Most often it is
sufficient to have a shared storage for source code but sometimes it
is better to have more than that.

%% Summarize the SCM feature
Each project has a repository which are fully backuped
daily. The developers can use the repository
freely except two predefined directories (pkg and www).


One can think of a situation where one has
written code, updated it and then decides that some junks of the
original code where better but are lost now. To overcome this one can
make use of version control. This is very important as in each stage
one can retrieve all historical code.

Further important stuff regarding collaborative development: easy
communication through various channels (forums, tracker, \ldots).

Advantages: useRs may participate and give feedback, larger software
projects can be better managed 


%% Rights management on R-Forge

%% FIXME: fluent change to rights management
Typically, users need to have read access to the data associated with
a certain project. Some of them (the developers) have write access to
the data. Usually there is a maintainer of the code. This person is the project
leader or has registered the project. Contributers can make use of the
provided facilities or email any changes to the code
that they developed---bug fixes, additional functionality---which the
maintainer adds to the code after verification.

%% Explaining rights management
R-Forge has its own rights management system. A registered user can be
granted one of several so called roles. To name a few there is
the \textit{Administrator} which typically are package
maintainers. The most important things only project administrators can do
are adding new users and releasing packages to CRAN directly in the R
tab. Other members of a project typically have either the role \textit{Senior
Developer} or \textit{Junior Developers}. Being granted one of these
rights means that you are permitted to commit to the project
SVN repository and examine 
the log files in the R tab.
When we speak of developers in subsequent sections we refer to project
members having th rights of at least of a \textit{junior developer}.

%% Quality Management and bug/support/feature tracker
Meaning that early version are typically prototypes and therefore are
not completely bug free. R-Forge can assist the developer with two
sort of tools. First it offers a qulaity management system similar to
that of CRAN. Packages on R-Forge are checked in a
standardized way based on \code{R CMD check} on different
platforms. The resulting logfiles can be examined by developers and
administrators  Second, 

%% FIXME: change to R tab
Of special interest for the R community is the possibility to have
daily prototype or development releases
of their packages. Furthermore, binaries for Windows and MacOS X are
also built on a daily basis. These packages can be
downloaded via a download link from the website or are installable
directly in R via \code{install.packages}. 
%%They are built from the ``pkg'' directory.

Every project can have an own project websites (ie.,
\code{http://packagename.R-Forge.R-project.org}). Basic html is
supported. It is checked out hourly from ``www'' directory of the
project's repository.

Project administrators are able to submit their current package
sources directly to CRAN over the web platform. 



%% sharing of documentation


%% WORD JUNKS AND PHRASES

%%%%%%%%%%%%%%%%%%%%%%%%%%%%%%%%%%%%%%%%%%%%%%%%%%%%%%%%%%%%%%%%%%%%%%%%
%% open source history and its advantages


%%The source code of open source software is by definition accessible to
%%users.  

%%Open source developers are said to work not for monetary returns, are
%%generally volunteers working together on a project. These members of
%%such a team may come from around the world and rarely meet.

%%This open source movement can be seen as an self-organizing process
%%which releases prototype code frequently which is reviewed by hundreds
%%of peers.   

%%One of the questions that may arise 
%%in this context is how to provide code to other developers
%%or even
%%more important how to collaborate with others to contribute to an open
%%source project. centralized resource for managing projects and
%%source code

%%As
%%collaborative development is a key factor for a lot of 
%%successfully open source projects. 

%%Open source development is
%%generally claimed to produce more bug free
%%code and to be available faster than closed source code. Open source
%%developers are said to work not for monetary returns, are generally
%%volunteers working together on a project. These members of such a
%%team may come from around the world and rarely meet.


%%This open source
%%movement can
%%be seen as an self-organizing process which releases prototype code
%%frequently which is reviewed by hundreds of peers.

%%Typically, users need to have read access to the data associated with
%%a certain project. Some of them (the developers) have write access to
%%the data. Usually there is a maintainer of the code. This person is the project
%%leader or has registered the project. Contributers can make use of the
%%provided facilities or email any changes to the code
%%that they developed---bug fixes, additional functionality---which the
%%maintainer adds to the code after verification.

%%%%%%%%%%%%%%%%%%%%%%%%%%%%%%%%%%%%%%%%%%%%%%%%%%%%%%%%%%%%%%%%%%%%%%%%
%% collaboration networks

%% Collaborative networks like social networks where the relationships
%% (connections between two persons) are collaborations. This
%% collaborations can be e.g., co-authoring a research paper or in this
%% case working together on a software project.
%% R-Forge tries to provide this decentralized research or software
%% projects a shared platform where developers can meet, join other
%% interesting projects, exchange knowledge and so forth. 

%%%%%%%%%%%%%%%%%%%%%%%%%%%%%%%%%%%%%%%%%%%%%%%%%%%%%%%%%%%%%%%%%%%%%%%%
%% GForge

%%Features provided by the open source GForge
%%system~\cite{manual:gforge} it can be obtained
%%from~\url{http://gforge.org} 
%%maintained and supported by the GForgeGroup
%%a php - postgresql framework for collaboration and source code management


\section{Howto get started}
%%A small guide through R-Forge---TODO: detailed description 

for more info see~\cite{theussl07:r_forge_users_manual}.

What has to be done in the first place to get started with an project
on R-Forge is to register a username on R-Forge. There is a link on
the main web site called ``New Account'' on the top right of the page.

Further things one can do on the main page are:

\begin{itemize}
\item login or create a new project
\item download the documentation
\item examine the latest news (also available as RSS feed)
\item search for projects either with the R-Forge's search
  functionality or through an 
  alphabetically listing using the 'Project Tree' tab. 
\item see some R-Forge specific statistics (how many projects/users,
  activity percentage, \ldots{})
\end{itemize}

After submitting the form (filled with your name, email address, and
some other information) an email has been sent to the given mail
address containing a link to activate your account. After you
activated your account you are able to join an existing project or
create a new one.

The latter is probably more of interest if you decided to migrate an
existing package to R-Forge and therefore we now explain the creation
of a project in more details. For joining an existing project we refer
to the user's manual.

\subsection{Registering a Project}

To achieve this either go to the main page and click on ``Register
Your Project'' or go to the ``My Page'' tab and click on ``Register
Project''. You are presented with a new form which you have to fill
out in order to register your project. Some remarks: Only the Project
Public Description will get visible on the project's homepage, whereas
the Project Purpose And Summarization only is an information for
us---the R-Forge administrators. The Project Unix Name is a name which
uniquely defines your project (e.g., it can be the name of your
package). There are several restrictions to it which are explained on this
page. E.g., according to the UNIX
filesystem conventions these names have to be in lower case (and will
be converted to it automatically in case you typed in upper case
characters).

After filling out the form and submitting it, the R-Forge
administrators decides whether this project gets approved or not. This
is only to prevent fake projects to get registered. After approval a
confirmation email is sent to the project administrator and the svn
repositories are created (note: this is done every hour on a fixed
time, so it is possible that you have to wait until you can check out
your repository).

To manage your project the administrator has several
possibilities. But first I want to explain what is important to know
about the project main page.

Every project has some sort of a own web area on R-Forge. Actually
there are two of them. A standardized one (which is the same for every
project on R-Forge) and a personalized one. The latter is accessible
through <Project Unix Name>.R-Forge.R-project.org and is managed via
the SVN repository (www directory). For useRs and developeRs the first
one may be more important and is therefore explained here in more
details.

On the Project Summary Page (typically the initial one when entering a
project) you see the following

\begin{itemize}
\item some details about the project (a short description,
  administrators, developers, \ldots{})

\item a download link which leads directly to the available packages
  of the project.
\item a link leading to the (personal) project homepage

\item links to activated features of the project (like tracker,
  Forums, Mailing Lists, \ldots{})

\item you'll find the latest news announcements (if made any)

\end{itemize}

In the project area each available feature is organized in a separate
tab. For example if you like to ask a question or discuss something
with the developers of the project you may click on the ``Forums'' tab
and enter the corresponding forum. The same is possible with mailing
lists. Actually, there is setup an initial mailing list called
<name>-commits which can be used to get commit message to the SVN
repository delivered (this has to be activated separately). New
mailing lists can be easily set up.

One of the most important tab is the R Packages tab. But before that
let us explain a bit the build process of R-Forge.

Every night a script parses the pkgs directories and exports the
packages found to a specific place. From this place the build machines
(currently a linux, mac und windows machine) sync these packages and
start to build them. Meanwhile the links to the source and the
binaries in the R packages tab get updated. After building is finished
the builds get synced back to R-Forge. They are available for download
then. Furthermore logs have been produced and are also available for
examination.

Another feature is the quality check we run every day. Check results
are also provided on the R tab.

%% some remarks to the build process



\section*{Outlook and future work}

%% 

\section{Acknowledgements}

Many thanks to Douglas Bates and the University of Wisconsin
for providing us with a server for hosting this platform
and. Furthermore, we have to thank the Computer Center 
of the Vienna University of Economics and Business Administration for
their support and for providing us with hardware as well as a
professional server infrastructure.

\bibliography{R-Forge}