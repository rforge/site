\title{Collaborative Software Development using R-Forge}
\author{Stefan Theu\ss{}l and Achim Zeileis}

\maketitle

\section*{Introduction}

%% Why should software developers use source code management tools?

%% Open source and its advantages
The source code of open source software is by definition accessible to
users. Open source development is claimed to produce more bug free
code, is available faster than closed source code. Open source
developers are said to work not for monetary returns, are generally
volunteers working together on a project. This team members often come
from around the world and rarely meet. This open source movement can
be seen as an self-organizing process which releases prototype code
frequently which is reviewed by hundreds of peers.  

%% collaboration networks

Collaborative networks like social networks where the relationships
(connections between two persons) are collaborations. This
collaborations can be e.g., co-authoring a research paper or in this
case working together on a software project.
R-Forge tries to provide this decentralized research or software
projects a shared platform where developers can meet, join other
interesting projects, exchange knowledge and so forth. 

%% version control
In science people want to work together and exchange knowledge and
this is also true when developing software. Most often it is
sufficient to have a shared storage for source code but sometimes it
is better to have more than that. One can think of a situation where one has
written code, updated it and then decides that some junks of the
original code where better but are lost now. To overcome this one can
make use of version control. This is very important as in each stage
one can retrieve all historical code.

Further important stuff regarding collaborative development: easy
communication through various channels (forums, tracker, \ldots).

Advantages: useRs may participate and give feedback, larger software
projects can be better managed 

%% difference maintainer, developers, users
Typically, users need to have read access to the data associated with
a certain project. Some of them (the developers) have write access to
the data. Usually there is a maintainer of the code. This person is the project
leader or has registered the project. Contributers can make use of the
provided facilities or email any changes to the code
that they developed---bug fixes, additional functionality---which the
maintainer adds to the code after verification.


%% What is GForge
GForge is a fork of the 2.61 SourceForge code. Since the codebase of
SourceForge has not been released over a certain time the GForge
project was formed. One of the authors of the original SourceForge
code, Tim Perdue, now maintains the GForge project.

%%important not only to the Open
%%Source community, but to the wider business community.

\section*{R-Forge}
%% what is R-Forge?
R-Forge (\url{http://R-Forge.R-project.org}) is a set of tools based
on the open source software GForge (\cite{manual:gforge})%% or: a system
 for collaborativi development of R Packages or R
related software. It provides source code management facilities
through Subversion (\cite{subversion07}) and various web-based features. 

R-Forge hosts nearly 100 Projects and has around 250 registered users at the
time of this writing.

%R-Forge is a tool which helps package developers to collaborate

%%Features provided by the open source GForge
%%system~\cite{manual:gforge} it can be obtained
%%from~\url{http://gforge.org} 
%%maintained and supported by the GForgeGroup
%%a php - postgresql framework for collaboration and source code management

%% not itemized but described in more details

Features for the R community are 


The most important feature is source code management with
Subversion. Each project has a repository which are fully backuped
daily. The developers can use the repository
freely except two predefined directories (pkg and www). %% connection
                                %% to below missing 

Interestng for the R community is the possibility to have daily built
of their packages for Unix/Linux (the source tarball), Windows
(32~bit binary) and MacOS X (universal binary). These packages can be
downloaded via a download link from the website or are installable
directly in R via install.packages. They are built from the ``pkg''
directory.

Every project can have an own project websites (ie.,
\url{http://package.R-Forge.R-project.org}). Basic html is
supported. It is checked out hourly from ``www'' directory of the
project's repository.

A file release system provides an
alternative way for submitting packages to CRAN. 

It is possible to have check results automatically delivered. 

%% sharing of documentation

%% A small guide through R-Forge---TODO: detailed description 

on the main page of R-Forge one can

\begin{itemize}

\item create an account, login, or create a new project

\item download the documentation

\item examine the latest news (also available as RSS feed)

\item search for projects either with the search function or through an
  alphabetically listing using the 'Project Tree' tab 

\item see some R-Forge statistics

\end{itemize}


on the project main page, so what can be done here?

\begin{itemize}

\item examine the project details

\item through the download link you get to the download page

\item there is also a link to the project homepage

\item developers and admins (maintainer) of the project are listed 

\item you'll find the latest news announcements

\end{itemize}


for more info see~\cite{theussl07:r_forge_users_manual}.

\section*{Summary}


\bibliography{R-Forge}