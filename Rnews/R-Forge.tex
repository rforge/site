\title{Collaborative Software Development Using R-Forge}
\author{Stefan Theu\ss{}l and Achim Zeileis}

\maketitle

%% Story of the article:
%% (1) Open source software (OSS) development - history, importance  _/
%%     - Apache, Linux famous examples ?TODO
%%     - Scripting languages, PHP, Perl and of course R ?TODO
%%     - software repositories: CPAN, CTAN -> CRAN ?TODO 
%% (2) collaboration and project management  _/
%%     - collaboration sites: SourceForge.net _/
%% (3) why is it important to have a central repository TODO
%%     - The Cathedral and the Bazaar TODO
%%     - large community -> better TODO
%% (4) - R-Forge  _/
%%     - explaining the basic tools  _/
%%     - R specific features  _/
%%       Package building/checking facilities
%%     - quality management through R check and tracker _/
%%     - other stuff: forums  _/

%% (5) How to forge packages on R-Forge (A day in a life of a new R
%%     package developer).
%%     - registering as a user  _/
%%     - registering a project (automatically creates SVN, a
%%       mailinglist -> managable through web interface -> click to send
%%       commit statements to this list).  _/
%%     - the svn base tree (picture of an axample tree with more
%%       packages) TODO


\section*{Introduction}

%% from useR abstract 
%% <useR>
%% Why should software developers use source code management tools?

A key factor in open source software (OSS) development is the rapid creation
of solutions within an open, collaborative environment. The open
source model had its major breakthrough with the increasing
usage of the internet. Online communities successfully combined
not only their programming effort but also their knowledge, work
and even their social life. OSS development is generally
claimed to be faster than closed source code because of the
``release early, release often'' paradigm introduced by Linus
Torvalds.
%% or: Additionally in the open source community it is appreciated to
%% release OSS more frequently than closed source software. This is
%% often called the ``release early, release often'' paradigm 
%% introduced by Linux Torvalds.
%% But this also means to manage the source code of a project.

%% TODO: following paragraph
Developers of larger projects like the language \R{} use source code
management (SCM) facilities to follow this paradigm.
The \R{} Development Core Team have been using
development tools like Subversion (SVN---see
(\citep{forge:Pilato+Collins-Sussman+Fitzpatrick:2004})) or 
Concurrent Versions System (CVS---see \citep{forge:Cederqvist:2006})
for managing their source code for a decade now.  
A central repository is hosted by ETH Z\"urich mainly for
managing the development of the base \R{} system. The
\R{}-project is now ready to provide similar infrastructure
for the entire \R{} community.

R-Forge (\url{http://R-Forge.R-project.org}) provides a set of tools
for source code management and various web-based
features. It aims to provide a platform for collaborative development of
\R{} packages, \R{}-related software or further projects.
All work is organized in a ``Project'' on R-Forge. 



%% brief overview of features possibly most interesting for readers
Furthermore, packages hosted on \R{}-Forge are built daily
for  various operating systems, namely Linux, MacOSX and Windows. These
package builds are downloadable from the website of the project on
\R{}-Forge as well as from a CRAN-style repository. This means that
packages are installable directly in \R{} via
\code{install.packages("foo", repos="http://R-Forge.R-project.org")}.

%% Article Outline
The present article is organized as follows. First, we present the core
features R-Forge is offering to the \R{} community. Second we
show how package developers can get started with 
\R{}-Forge. In particular we show how people can register a project, 
use \R{}-Forge's source code management facilities, provide their
packages with \R{}-Forge, host a project specific website, and
finally submit a package to CRAN (\url{CRAN.R-project.org}). Then ...
Eventually we summarize ... and give a brief outlook future work.


%% What is R-Forge offering to you?
%% Alternative title: Core Features of R-Forge
\section{R-Forge}
%% what is R-Forge?
R-Forge (\url{http://R-Forge.R-project.org}) is a set of tools for
collaborative development of \R{} packages, \R{}
related software or other projects which are somehow related to the
language \R{}.

It is based on the OSS GForge (\citep{forge:copeland_et_al:2006}) which is a
fork of the 2.61 SourceForge code maintained by one of the authors of
the original SourceForge code Tim Perdue. GForge provides source code
management facilities through SVN and various web-based features
accessible via so called ``tabs''. Tabs can be distinguished in main
tabs (currently Home, My Page and Project Tree) and project specific
tabs which are only visible when you access a project.

%% Additional Information to GForge. Since the codebase of 
%% SourceForge has not been released over a certain time the GForge
%% project was formed. 

%% Why should software developers use source code management tools?
%% FIXME: How to formulate better
%% Version control - probably the most important feature

Probably the most important features correspond to 3 project tabs are
``Summary'' which offers an overview of the whole project, ``SCM''
describing how people can access the source code repository and ``R
Packages'' containing a list of packages available in this Project.

%% eventually include history of the development of R-Forge
R-Forge hosted nearly 200 Projects and had around 450
registered users in its development stage. This and the steadily growing list of
feature requests show that there is a high demand for centralized source code
management tools and for releasing prototype code frequently among the
\R{} community.

\subsection{Source Code Management}

Why is source code management important? Especially in science people
work together and exchange knowledge and 
this is also true for developing software as well. Most often it is
sufficient to have a shared storage for source code but sometimes it
is better to have more than that. For example, version control enables
a group of developers to work on a given project simultaneously
without thinking about details like merging different code junks
together or throwing away code as there is always the possibility to
go back to a specific version of the whole software repository.

%% Summarize the SCM feature
On R-Forge each project has such a version controlled repository
including all features provided by SVN. In addition to the
backup inherent backup of every version in an SVN repository a backup
of the whole repository is made daily. 
%% One can think of a situation where one has
%% written code, updated it and then decides that some junks of the
%% original code where better but are lost now. To overcome this one can
%% make use of version control. This is very important as in each stage
%% one can retrieve all historical code.

%% except two predefined directories (\code{pkg} and \code{www})

%% Rights management on R-Forge

Typically, users need to have read access to the data associated with
a certain project. Some of them (the developers) have write access to
the data. Usually there is a maintainer of the code. This person is the project
leader, package maintainer or put simply has registered the
project. 
%% Explaining rights management
To achieve this R-Forge has its own rights management system. A
registered user can be 
granted one of several so called roles. To name a few there is
the \textit{Administrator} which typically are the package
maintainers or project leaders. For example, only  administrators can
add new users 
to the project or release packages directly to CRAN using the ``R Packages''
tab. Other members of a project typically have either the role \textit{Senior
Developer} or \textit{Junior Developers}. Being granted one of above
rights means that you are permitted to commit to the project
SVN repository and examine the log files in the ``R Packages'' tab.
When we speak of developers in subsequent sections we refer to project
members having the rights of at least of a \textit{Junior Developer}.

%% Contributers can make use of the 
%% provided facilities or email any changes to the code
%% that they developed---bug fixes, additional functionality---which the
%% maintainer adds to the code after verification. How is this organized
%% in R-Forge? 

%%
%%
\subsection{Release and Quality Management of R Packages}

%% Quality Management and bug/support/feature tracker
Meaning that early version are typically prototypes and therefore are
not completely bug free. R-Forge can assist the developer with two
sort of tools. First it offers a quality management system similar to
that of CRAN. Packages on R-Forge are checked in a
standardized way based on \code{R CMD check} on different
platforms. The resulting log files can be examined by developers.

Second, especially of interest for the \R{} community is the
possibility to have daily prototype or development releases
of their packages. Furthermore, binaries for Windows and MacOS X are
also built on a daily basis. These packages can be
downloaded via a download link in the ``R Packages'' tab or
are directly installable in \R{} using the function \code{install.packages()}. 
%%They are built from the ``pkg'' directory.

\subsection{Additional Features}

Additionally, \textbf{R-Forge} provides other tools especially for
communication between project members and their user base. Further
tools help developers of larger projects to coordinate their work.
What follows is a brief summary of these tools.
\begin{itemize}
\item Project websites are a way for developers to present their work on a
subdomain of \url{http://r-forge.r-project.org}. It is also allowed to
have a link on the project summary page to another website. 
%% checked out hourly from ``www'' directory of the
%%project's repository.
\item Mailing Lists are nowadays valuable for every kind of project.By default one is automatically created when setting up a project. Additional mailing lists can be
  created rather easily. 
\item The project tree offers you to assign your project to one or more
  topics (e.g.: biostatistics, finance, regression analysis,
  \ldots). This helps other people to quickly find what they are
  looking for. In the default setting the project tree lists
  alphabetically all
  projects including a short description similar to CRAN.
\item Forums can be set up separately by the project
  administrators.%% are places to discuss certain topics 
\item Bug tracking and feature request systems assist in package
  development process.
\item News can be put on the project summary page as well as on the
  front page. The latter needs an approval by one of the R-Forge
  administrators). It is possible to download them as RSS feeds.
\end{itemize}

\section{How to get started}
%%A small guide through R-Forge---TODO: detailed description 
In this section we show how new users can get started with
R-Forge. For a more detailed guide to R-Forge and additional
information we refer to the user's manual
\cite{forge:theussl:2008}.

If you type the URL \url{http://R-Forge.R-project.org} into your
browser you get to the main page. Here you have the following
opportunity to
\begin{itemize}
\item login,
\item register a user or a project,
\item download the documentation,
\item examine the latest news (also available as RSS feed),
\item enter a specific project website either by search for available
  projects using the search field on the top middle of the page, going
  through an alphabetically listing seen in the
  ``Project Tree'' tab or by clicking on one of the projects listed on
  the right of the front page,
\item and examine R-Forge specific statistics (how many projects/users
  are currently registered on R-Forge, project activity percentages,
  \ldots{}).
\end{itemize}

There is another tab on the front page named ``My Page'', where you
can configure your account. But before you can use it, you have to
register yourself as a user.

\subsection{Registering as a New User}

What has to be done in the first place to get started with a project
on R-Forge is to register yourself as a user on R-Forge. There is a link on
the main web site called ``New Account'' on the top right of the page
which leads to the registration form.

After submitting the form (filled with your name, email address, and
some other information) an email gets sent to the given mail
address containing a link for activating your account. After you
activated your account you are able to join an existing project or
create a new one.
The latter is probably more of interest if you decide to migrate an
existing package to R-Forge. Therefore we explain the registration 
of a project in more details. For joining an existing project we refer
to the user's manual.

\subsection{Registering a Project}

To achieve this either go to the main page and click on ``Register
Your Project'' or go to the ``My Page'' tab and click on ``Register
Project''. You are presented with a new form which you have to fill
out in order to register your project. Some remarks: Only the Project
Public Description will get visible on the project's homepage, whereas
the Project Purpose And Summarization only is an information for
us---the R-Forge administrators. The Project Unix Name is a name which
uniquely defines your project (e.g., it can be the name of your
package). There are several restrictions to it which are explained on this
page. E.g., according to the UNIX
file system conventions these names have to be in lower case (and will
be converted to it automatically in case you typed in upper case
characters).

After filling out the form and submitting it, the R-Forge
administrators decides whether this project gets approved or not. This
is only to prevent fake projects to get registered. After approval a
confirmation email is sent to the project administrator and the svn
repositories are created (note: this is done every hour on a fixed
time, so it is possible that you have to wait until you can check out
your repository).

To manage your project the administrator has several
possibilities. But first I want to explain what is important to know
about the project main page.

Every project has some sort of a own web area on R-Forge. Actually
there are two of them. A standardized one (which is the same for every
project on R-Forge) and a personalized one. The latter is accessible
through <Project Unix Name>.R-Forge.R-project.org and is managed via
the SVN repository (www directory). For useRs and developeRs the first
one may be more important and is therefore explained here in more
details.

On the Project Summary Page (typically the initial one when entering a
project) you see the following

\begin{itemize}
\item some details about the project (a short description,
  administrators, developers, \ldots{})

\item a download link which leads directly to the available packages
  of the project.
\item a link leading to the (personal) project homepage

\item links to activated features of the project (like tracker,
  Forums, Mailing Lists, \ldots{})

\item you'll find the latest news announcements (if made any)

\end{itemize}

In the project area each available feature is organized in a separate
tab. For example if you like to ask a question or discuss something
with the developers of the project you may click on the ``Forums'' tab
and enter the corresponding forum. The same is possible with mailing
lists. Actually, there is setup an initial mailing list called
<name>-commits which can be used to get commit message to the SVN
repository delivered (this has to be activated separately). New
mailing lists can be easily set up.

One of the most important tab is the R Packages tab. But before that
let us explain a bit the build process of R-Forge.

Every night a script parses the pkgs directories and exports the
packages found to a specific place. From this place the build machines
(currently a linux, mac und windows machine) sync these packages and
start to build them. Meanwhile the links to the source and the
binaries in the R packages tab get updated. After building is finished
the builds get synced back to R-Forge. They are available for download
then. Furthermore logs have been produced and are also available for
examination.

Another feature is the quality check we run every day. Check results
are also provided on the R tab.

%% some remarks to the build process

\section*{Recent and future developments}
In this section we summarize the developments which took place between
the first release basically consisting of the features of GForge and
the current release Version. Then we briefly give an outlook to future
developments.

Recently added features and major changes include:
\begin{itemize}
\item Enhanced structure in the SVN repository allows multiple
  packages in one project. Each package is a subdirectory of the
  \code{/pkg} directory.
\item An additional check box on the news submit page helps to
  separate front page news from project-only news. Nevertheless,
  whenever a news item is declared to be put on the front page,
  R-Forge administrators need to approve them (default project-only
  submission).
\item A new button in the SCM Admin tab allows for delivery of SVN
  commit messages (default: off).  This
  list is used for circulating SVN commit messages to registered
  users (has to be turned on separately).
\item In newly registered projects by default only the SCM, the R
  packages tab and mailing lists are turned on. Forums, News and
  Tracker can be turned on separately using ``Edit Public Info'' in
  the Admin tab of the project. This is aimed to help new users to get
  in touch with R-Forge and to let users decide which features they
  want to use.
\item We moved to a platform independent package building and quality
  management code combined in one \R{} package.
\end{itemize}

Currently on the wishlist and under development:
\begin{itemize}
\item Wiki,
\item Task management,
\item Restructured tracker more compatible to \R{}. E.g., \R{} version
  used which generated the bug.  
\end{itemize}

%% - 

\section{Acknowledgements}

Setting up this project would not have been possible without Douglas
Bates and the University of Wisconsin 
They provided us with a server for hosting this platform. Furthermore,
the authors would like to thank the Computer Center 
of the Wirtschaftsuniversit\"at Wien for
their support and for providing us with hardware as well as a
professional server infrastructure.

\bibliography{R-Forge}