\title{Collaborative Software Development using R-Forge}
\author{Stefan Theu\ss{}l and Achim Zeileis}

\maketitle

\section*{Introduction}

%% Why should software developers use source code management tools?

%% Open source and its advantages
The source code of open source software is by definition accessible to
users. Open source development is generally claimed to produce more bug free
code and to be available faster than closed source code. Open source
developers are said to work not for monetary returns, are generally
volunteers working together on a project. These members of such a
team may come from around the world and rarely meet.

This open source
movement can
be seen as an self-organizing process which releases prototype code
frequently which is reviewed by hundreds of peers.  

%% collaboration networks

Collaborative networks like social networks where the relationships
(connections between two persons) are collaborations. This
collaborations can be e.g., co-authoring a research paper or in this
case working together on a software project.
R-Forge tries to provide this decentralized research or software
projects a shared platform where developers can meet, join other
interesting projects, exchange knowledge and so forth. 

%% version control
In science people want to work together and exchange knowledge and
this is also true when developing software. Most often it is
sufficient to have a shared storage for source code but sometimes it
is better to have more than that. 


One can think of a situation where one has
written code, updated it and then decides that some junks of the
original code where better but are lost now. To overcome this one can
make use of version control. This is very important as in each stage
one can retrieve all historical code.

Further important stuff regarding collaborative development: easy
communication through various channels (forums, tracker, \ldots).

Advantages: useRs may participate and give feedback, larger software
projects can be better managed 

%% difference maintainer, developers, users
Typically, users need to have read access to the data associated with
a certain project. Some of them (the developers) have write access to
the data. Usually there is a maintainer of the code. This person is the project
leader or has registered the project. Contributers can make use of the
provided facilities or email any changes to the code
that they developed---bug fixes, additional functionality---which the
maintainer adds to the code after verification.


%% What is GForge !!!! CITATION !!!!
GForge is a fork of the 2.61 SourceForge code. Since the codebase of
SourceForge has not been released over a certain time the GForge
project was formed. One of the authors of the original SourceForge
code, Tim Perdue, now maintains the GForge project.

%%important not only to the Open
%%Source community, but to the wider business community.

\section*{R-Forge}
%% what is R-Forge?
R-Forge (\url{http://R-Forge.R-project.org}) is a set of tools based
on the open source software GForge (\cite{manual:gforge})%% or: a system
 for collaborative development of \proglang{R} Packages, \proglang{R}
related software or other projects which are somehow related to our
all beloved language \proglang{R}. It provides source code management facilities
through Subversion (\cite{subversion07}) and various web-based features. 

R-Forge hosts nearly 150 Projects and has around 350 registered users at the
time of this writing.

%R-Forge is a tool which helps package developers to collaborate

%%Features provided by the open source GForge
%%system~\cite{manual:gforge} it can be obtained
%%from~\url{http://gforge.org} 
%%maintained and supported by the GForgeGroup
%%a php - postgresql framework for collaboration and source code management

%% not itemized but described in more details


%% Summarize features
Features for the R community are 

The most important feature is source code management with
Subversion. Each project has a repository which are fully backuped
daily. The developers can use the repository
freely except two predefined directories (pkg and www). %% connection
                                %% to below missing 

Interestng for the R community is the possibility to have daily built
of their packages for Unix/Linux (the source tarball), Windows
(32~bit binary) and MacOS X (universal binary). These packages can be
downloaded via a download link from the website or are installable
directly in R via install.packages. They are built from the ``pkg''
directory.

Every project can have an own project websites (ie.,
\url{http://package.R-Forge.R-project.org}). Basic html is
supported. It is checked out hourly from ``www'' directory of the
project's repository.

A file release system provides an
alternative way for submitting packages to CRAN. 

It is possible to have check results automatically delivered. 

%% sharing of documentation
\section{Howto get started}
%%A small guide through R-Forge---TODO: detailed description 

What has to be done in the first place to get started with an project
on R-Forge is to register a username on R-Forge. There is a link on
the main web site called ``New Account'' on the top right of the page.

Further things one can do on the main page are:

\begin{itemize}
\item login or create a new project
\item download the documentation
\item examine the latest news (also available as RSS feed)
\item search for projects either with the R-Forge's search
  functionality or through an 
  alphabetically listing using the 'Project Tree' tab. 
\item see some R-Forge specific statistics (how many projects/users,
  activity percentage, \ldots{})
\end{itemize}

After submitting the form (filled with your name, email address, and
some other information) an email has been sent to the given mail
address containing a link to activate your account. After you
activated your account you are able to join an existing project or
create a new one.

The latter is probably more of interest if you decided to migrate an
existing package to R-Forge and therefore we now explain the creation
of a project in more details. For joining an existing project we refer
to the user's manual.

\section{Registering a Project}

To achieve this either go to the main page and click on ``Register
Your Project'' or go to the ``My Page'' tab and click on ``Register
Project''. You are presented with a new form which you have to fill
out in order to register your project. Some remarks: Only the Project
Public Description will get visible on the project's homepage, whereas
the Project Purpose And Summarization only is an information for
us---the R-Forge administrators. The Project Unix Name is a name which
uniquely defines your project (e.g., it can be the name of your
package). There are several restrictions to it which are explained on this
page. E.g., according to the UNIX
filesystem conventions these names have to be in lower case (and will
be converted to it automatically in case you typed in upper case
characters).

After filling out the form and submitting it, the R-Forge
administrators decides whether this project gets approved or not. This
is only to prevent fake projects to get registered. After approval a
confirmation email is sent to the project administrator and the svn
repositories are created (note: this is done every hour on a fixed
time, so it is possible that you have to wait until you can check out
your repository).

To manage your project the administrator has several
possibilities. But first I want to explain what is important to know
about the project main page.

Every project has some sort of a own web area on R-Forge. Actually
there are two of them. A standardized one (which is the same for every
project on R-Forge) and a personalized one. The latter is accessible
through <Project Unix Name>.R-Forge.R-project.org and is managed via
the SVN repository (www directory). For useRs and developeRs the first
one may be more important and is therefore explained here in more
details.

On the Project Summary Page (typically the initial one when entering a
project) you see the following

\begin{itemize}
\item some details about the project (a short description,
  administrators, developers, \ldots{})

\item a download link which leads directly to the available packages
  of the project.
\item a link leading to the (personal) project homepage

\item links to activated features of the project (like tracker,
  Forums, Mailing Lists, \ldots{})

\item you'll find the latest news announcements (if made any)

\end{itemize}

In the project area each available feature is organized in a separate
tab. For example if you like to ask a question or discuss something
with the developers of the project you may click on the ``Forums'' tab
and enter the corresponding forum. The same is possible with mailing
lists. Actually, there is setup an initial mailing list called
<name>-commits which can be used to get commit message to the SVN
repository delivered (this has to be activated separately). New
mailing lists can be easily set up.

One of the most important tab is the R Packages tab. But before that
let us explain a bit the build process of R-Forge.

Every night a script parses the pkgs directories and exports the
packages found to a specific place. From this place the build machines
(currently a linux, mac und windows machine) sync these packages and
start to build them. Meanwhile the links to the source and the
binaries in the R packages tab get updated. After building is finished
the builds get synced back to R-Forge. They are available for download
then. Furthermore logs have been produced and are also available for
examination.

Another feature is the quality check we run every day. Check results
are also provided on the R tab.

%% some remarks to the build process

for more info see~\cite{theussl07:r_forge_users_manual}.

\section*{Outlook}

Many things are still on our wishlist but not yet implemented. Be it 

\section{Acknowledgements}

Many thanks to the Douglas Bates and the University of Wisconsin
for providing us with a server for hosting this platform and one of
the build machines. Furthermore, we have to thank the Computer Center
of the Vienna University of Economics and Business Administration for
their support and for providing us with hardware as well as a
professional server infrastructure.

\bibliography{R-Forge}