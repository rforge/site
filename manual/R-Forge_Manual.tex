\documentclass[a4paper]{article}

% PREAMBLE
%\usepackage[austrian]{babel}
%\usepackage[utf8]{inputenc}
\usepackage[authoryear,round]{natbib}
\usepackage{graphicx}
\usepackage{hyperref}
\usepackage{url}
\usepackage{a4wide}
\sloppy

%% Authornames: R-Forge Team, R-Forge Administration Team, R-Forge
%% Support Team, R-Forge Maintenance Team, synonym for team -> crew
\newcommand{\RFORGE}{R-Forge Administration and Development Team}

\title{R-Forge User's Manual, BETA}
\author{\RFORGE}
%\email{R-Forge@R-project.org}
%\institute{Department of Statistics and Mathematics\newline
%Vienna University of Economics and Business Administration}
\date{\today}

\let\code=\texttt
\let\email=\texttt
\newcommand{\pkg}[1]{{\normalfont\fontseries{b}\selectfont #1}}
\newcommand{\proglang}[1]{\textsf{#1}}
\newcommand{\class}[1]{`\code{#1}'}

\begin{document}

\pagestyle{empty}

%%\maketitle

\begin{titlepage}

  {\LARGE \raggedleft \textbf{R-Forge User's Manual, {\large BETA}}}\\
%%  \vspace*{0.5cm}
  \rule{\linewidth}{1.5mm}
  \begin{flushright}
  \textbf{SVN Revision: 47, \today}\\
  \end{flushright}

  \vspace*{\fill}
%%  \vspace*{10cm} 
   
  {\Large
    \noindent {\textbf{\RFORGE{}}\\
    \rule{\linewidth}{1mm}} 
  }
  
\end{titlepage}


\newpage
\vspace*{\fill}

%% Copyright notices
%%\noindent Copyright 2006--2007 Stefan Theussl\\
\noindent Copyright 2006--2008 \RFORGE{}\\

\vspace{0.5cm}
\includegraphics{CCAL.png} 
\vspace{1cm}

\noindent The content in this manual is licensed
under a Creative Commons Attribution-Share Alike 2.0 license. 

\vspace{0.5cm}
The \RFORGE{} has chosen to apply the Creative
Commons Attribution License (CCAL) to this manual, i.e., that under the CCAL,
the authors retain ownership of the copyright 
for this manual, but the authors allow anyone to download, reuse,
reprint, modify, distribute, and/or copy the contents of this manual,
so long as the original authors and source are 
credited. Furthermore, the authors permit others to distribute
derivative works only under the same license or one compatible with
the one that governs the authors' work. 
This broad license was developed to facilitate open access
to, and free use of, original works of all types. Applying this
standard license to this work will ensure the authors' right to make
this work freely and openly available (see
\url{http://creativecommons.org/licenses/by-sa/2.0/} for details). 

\vspace{0.5cm}
The current members of the \RFORGE{} are Kurt Hornik, Martin Kober,
David Meyer, Stefan Theu\ss{}l and Achim Zeileis. To contact the
authors please write an email to \email{R-Forge@R-project.org}.

\newpage

\pagenumbering{roman}
\pagestyle{plain}
\tableofcontents

\clearpage
\pagestyle{headings}
\pagenumbering{arabic}
\setcounter{page}{1}

\section{Introduction}
\label{sec:intro}

For a decade the \proglang{R} Development Core Team is using collaborative
development tools like Subversion~\citep[SVN,
see][]{forge:Pilato+Collins-Sussman+Fitzpatrick:2004} or Concurrent
Versions System~\citep[CVS, see][]{forge:Cederqvist:2006} provided by
ETH Z\"urich.   
Many \proglang{R} package
developers around the world use similar infrastructure built upon their own
solutions as collaborative development is a key factor for a lot of
successfull open source projects. Therefore the \textbf{R-project}
wants to provide infrastructure not only
for the \proglang{R} Development Core Team but also for the entire
\proglang{R} community. 

R-Forge~(\url{http://R-Forge.R-project.org}) provides a set of tools
for source code management and various web-based
features. It aims to provide a platform for collaborative development of
\proglang{R}~packages, \proglang{R}-related software or further projects. It is
closely related to the most famous of such platforms---the 
world's largest open source software (OSS) development website---namely
\url{http://SourceForge.net}. What is different is the provided set of
tools with a strong focus on the need of the \proglang{R}
community. All in all, R-Forge is a place where all \proglang{R}
developers and users can come together and exchange 
their knowledge. Joining this community service gives you access to
all we have to offer. 

Basically, R-Forge is based on 
GForge~\citep{forge:copeland_et_al:2006} which is a framework
integrating various tools like 
SVN for collaborative work on source code, mailing lists, bug tracking
etc. into one platform. On R-Forge all work is organized in
the same entity namely a ``Project''. In these projects it is possible
to host one or more packages. The main advantage is, that your package
is going to be built and checked every day 
not only with the latest patched version of \proglang{R} but also with
the release candidate just before a new version of \proglang{R} is to be
released. It is not necessary anymore, that developers have to set up
a local platform-independend build and check environment. Furthermore,
packages passing the quality management system on R-Forge can be
released to CRAN in a standardized way.

Additionally, R-Forge provides other tools especially for larger
projects to coordinate the work between project members and to
communicate with their user base.

\begin{itemize}
\item Project websites are a way for developers to present their work
  on a subdomain of R-Forge (E.g.,
  \url{http://foo.R-forge.R-project.org}). It is also possible to
  offer a link to another website on the project summary page instead. 
\item Mailing lists: By default one is automatically created when setting up a
  project. Additional mailing lists can be set up easily as well. 
\item Projects can be categorized into different
  topics (e.g., biostatistics, finance, regression analysis,
  \ldots). This enables other people to quickly find what they are
  looking for. People can browse the categories in the so-called
  ``Project Tree'' tab. In the default setting the project tree lists
  alphabetically all projects including a short description similar to
  CRAN. It is clear that this tree cannot be complete, so people are
  welcome to make suggestions on how we could improve it. 
\item Forums can be set up separately by the project
  administrators.%% are places to discuss certain topics 
\item News can be submitted to the project summary page as well as to the
  front page. The latter needs approval by one of the \RFORGE{}
  members. It is also possible to download them as RSS feeds.
\end{itemize} 

%%Upcoming features:
%%\begin{itemize}
%%\item Wiki,
%%\item task management
%%\item code snippets,
%%\item project help board.
%%\end{itemize}

In this manual we present all relevant steps to get
started with R-Forge. If you need further help or have
comments regarding R-Forge please send an email to
\email{R-Forge@R-project.org}. For a more detailed
documentation regarding the underlying GForge system
see~\cite{forge:copeland_et_al:2006}. 

\section{Registration}
\label{sec:registration}

\subsection{Registering a  New User}

To register a new user, click on the ``New Account'' link on the top
right side of the browser window at \url{http://R-Forge.R-project.org}.
Fill out the form (there are descriptions and hints for each field on
this site) and click on ``Submit'' afterwards. You will receive an
email with
a URL to your specified email address. After clicking on this link
your account has been verified. Now you're able to login to the
website.
\newline

\textbf{Important note:} The tab ``My Page'' is the most important
page on R-Forge. Here you configure your account, you see your
project memberships and see the items, which have been assigned to you
(i.e. bugs, feature requests, etc.).
\newline

Now you can start your own project (see Section~\ref{sec:newproject}
for details) or become a member of an existing project
(Section~\ref{sec:joinproject}). 

\subsection{Joining a Project}
\label{sec:joinproject}
If you like to join an existing project you achieve this by doing the
following steps:
\begin{enumerate}
\item First you need the name of the project you want to become a
  member of. You can ``search'' for the project (top middle side of the
  browser window) or you click on one of the projects showing in the
  ``Project Tree''. The latter is an alphabetically sorted list of all
  \textbf{R-Forge} projects.
\item Then you go to the project summary page (should be the default
  entry point). There is a window on the right side called
  ``Developer Info''. To join this project you need the permission of
  the project admin. So click on ``Request to join'' to send the
  project admin an email.
\item If the project admin decides to add you as developer you will
  receive an email confirming your developer account. Now you have
  full SVN access (see section \ref{sec:scm} for details).
\end{enumerate}

\subsection{Registering a New Project}
\label{sec:newproject}

Registering a new project is easy: Go to the R-Forge website, login and
go to ``My Page'' section. You have a ``Register Project'' link in the
menu at the top of your page. Now fill in the form and submit your
project. After approval of the project by the R-Forge admins  you will
be notified via email and you will be able to start with your project
on R-Forge.

\section{Source Code Management}
\label{sec:scm}

\subsection{How to get SVN to work}
\label{sec:scmhowto}

\textbf{R-Forge} uses
\textbf{Subversion}~(SVN, \url{http://subversion.tigris.org}) for
source code management.
Therefore you need an SVN client (e.g. ``Tortoise SVN'' on
windows machines,~\url{http://tortoisesvn.tigris.org}). For security
reasons we use secure shell (SSH) tunneling for
developer accounts which means that all network traffic is
encrypted. Therefore you have to setup your machine accordingly.

\subsubsection{Windows}

The software mentioned in this section can be found
on\newline
\url{http://www.chiark.greenend.org.uk/~sgtatham/putty/}.
\begin{enumerate}
\item get and install the latest TortoiseSVN client from
  \url{http://tortoisesvn.net/downloads}.
\item it is sufficient to use password authentication to write to the
  project repository (if you decided to do so please go to
  step~5). But for convenience purposes (avoiding password 
  queries) it is probably better to
  generate an SSH keypair and upload the \textbf{public} 
  key to \textbf{R-Forge}:
  \begin{itemize}
  \item Download and execute \texttt{puttygen.exe}.
  \item leave ssh-2 rsa marked and click on ``generate'' (make some
    random movements).
  \item save the private key using the corresponding button (you don't
    need to set a password).
  \item  Mark the text in the text field ``Public key
    for pasting into OpenSSH authorized\_keys file''.
  \item copy and paste it into your shell account information
    configuration: Go to \url{http://R-forge.R-project.org} and
    login. Go to ``My Page'' and then click on ``Account
    maintenance''. At the bottom 
    of this page you click on ``edit keys'' in the ``Shell Account
    Information'' window (\textbf{Important note:} you must be a project admin
    or member of a project to do this, otherwise there won't be an
    option ``Shell Account Information'', see section
    \ref{sec:joinproject} for joining a project or section
    \ref{sec:newproject} for registering a project).
    The key you enter here is typically of the form:
    \begin{verbatim}
    ssh-rsa AAAA... foo@bar
    \end{verbatim}
    The first field
    describes the type of key, the second field is the key itself, and
    the third field is a comment. \textbf{It is important that there are no
      newlines within a key}.
  \end{itemize}
\item Now you have to wait until the next full hour. The keys are
  activated once an hour only.
\item Next you need an authentication agent like \texttt{pageant.exe}. Load
  your \textbf{private} key with \texttt{pageant.exe} (right click on
  the pageant tray icon and then ``add key''.
%%\item Before you do your first checkout you need the host
%%  key information of R-Forge. Therefore you fetch the registry key
%%  from
%%  \url{http://download.r-forge.r-project.org/r-forge-hostname.reg} and
%%  add it to the registry by double clicking on the file.
\item Finally check out the repository using the URL given on the
  project website (tab ``SCM'') under ``developer account'' with
  ``Tortoise SVN'' or the SVN~client of your choice. (Note: if this last step
  fails for some reason, wait an hour an try again. The logins are
  created only once an hour only)
\end{enumerate}

\subsubsection{Linux/Unix}
\label{sec:scm-unix}

\begin{enumerate}
\item It is sufficient to use password authentication to write to the
  project repository (if you decided to do so please go to
  step~4). But for convenience purposes (avoiding password 
  queries) it is probably better to upload an SSH keypair: Generate
  and save the keys 
  using \texttt{ssh-keygen} on the command line or use your existing keypair.  
\item Upload the \textbf{public} key to \textbf{R-Forge} using the webplatform: Go to
  ``My Page'' and then click on ``Account maintenance''. At the bottom
  of this page you click on ``edit keys'' in the ``Shell Account
  Information'' window (\textbf{Important note:} you need to be a project admin
  or member of a project to do this, otherwise there won't be an
  option ``Shell Account Information'', see section
  \ref{sec:joinproject} for joining a project or section
  \ref{sec:newproject} for registering a project).
  The key you enter here is typically of the form:
\begin{verbatim}
ssh-dsa AAAA... foo@bar
\end{verbatim}
  The first field
  describes the type of key, the second field is the key itself, and
  the third field is a comment. \textbf{It is important that there are
    no newlines within a key}.
\item Now you have to wait until the next full hour (the key gets activated).
\item Finally check out the repository using the URL given on the
  project website (tab ``SCM'') under ``developer account'' using
  \texttt{svn checkout}. (Note: if this last step
  fails for some reason, wait an hour and try again. The logins are
  created once an hour only)
\end{enumerate}



\subsubsection{Mac OS}
\label{sec:scm-macosx}
%% Thanks to Philippe Grosjean for providing us with this description
Mac~OS users should refer to Unix section~\ref{sec:scm-unix}. They
can alternatively use GUI~applications to manage their SVN
repositories and SSH~keychains.

\begin{enumerate}
\item To manage the ssh keypair you can use SSHKeychain
(\url{http://www.sshkeychain.org/}). You can generate your keychains
in the 'Preferences...' dialog box of SSHKeychain, but you have to provide a
passphrase of 5 characters minimum. If you do not want a passphrase,
use 'ssh-keygen' at the terminal (a keychain without passphrase is
required to use SCPlugin, see step~4. In SSHKeychain, select 'Agent'
- 'Add single key...' and select the key you just generated. 

\item Upload the public key to \textbf{R-Forge} as explained in
  section~\ref{sec:scm-unix}
  Unix. The public key is in a file name [keyfile].pub in the same
  directory as your private key. If you used default directory and
  name to generate your key, the files are in a hidden directory named
  '.ssh' in your home path, and is thus not visible in the Finder. To
  make a copy of the public key on your desktop type the following
  commands in the terminal (replace the name of the file with yours): 

  \code{cp ~/.ssh/id\_rsa.pub ~/Desktop/id\_rsa.pub}
  
  Then, upload this file to \textbf{R-Forge}.

\item Wait until the next full hour so that the key gets activated.

\item If you want a convenient management of your SVN~repository in the
  Finder, install SCPlugin. To install it, follow instructions at
  \url{http://scplugin.tigris.org/installation.html}. Once SCPlugin is
  installed, navigate in the Finder to the directory where you want to
  place the local copy of your SVN~repository. Create a folder with the name
  of your project, right click on this folder and select \textit{Subversion}
  -> \textit{Checkout}. Indicate the URL given on the project website (tab
  "SCM") under "developer account". (Note: if this last step fails,
  wait an hour and try again. The logins are created once an hour
  only). 

\end{enumerate}

Another page to consult for SVN installation on Mac~OS is
\url{http://www.wikihow.com/Install-Subversion-on-Mac-OS-X}. Alternate SVN
software are proposed there. 

\section{R Package Development}

R-Forge aims to provide a platform for collaborative development of
\proglang{R}~packages, \proglang{R}-related software or further
projects. Especially providing good infrastructure for the development of
\proglang{R}~packages is a main mission of the \RFORGE{}. Packages
committed to the SVN repositories are built and checked on multiple
plattforms, including Linux, Windows (32 bit) and Mac~OS. 

\subsection{Building and Checking}
To build your packages, simply place your package or, alternatively,
multiple packages in the
\textbf{\texttt{pkg/}} directory in your SVM repository (see 
Section~\ref{sec:scmhowto} on how to achieve this) and check it
in. Typically this directory contains either the R package with the usual
\texttt{DESCRIPTION} file and \texttt{R/}, \texttt{man/},
\texttt{data/} directories (see \cite{Rcore:writing_R_extensions} for more
details) or contains two or more directories containing the actual
package sources, i.e., you can build \textbf{more than one package} by
putting the packages in subdirectories, e.g. \texttt{pkg/foo/}, 
\texttt{pkg/bar/}, etc.

Packages are automatically checked out every night (Central European
Time), built and checked on several plattforms (make sure that the
\texttt{DESCRIPTION} files are well-formed and contain a proper 
package name, otherwise your package may not get checked out). All
packages successfully checked out appear on the project home page in
the \proglang{R} packages tab. Project members also have access to the
\textbf{check and build logs} on all supported plattforms.  

Furthermore, these packages are made available in a CRAN-style
repository---the
R-Forge package repository. This allows packages to be installed simply by
typing \texttt{install.packages("foo", repos = "http://R-Forge.R-project.org")}
on the \proglang{R} command prompt. 

\subsection{External SVNs}
Developers who wish to use their own external package repository can
use the \textbf{external SVN} feature in the R~packages Admin~tab
(follow the instructions on the site). Please note that only
repositories with anonymus access and using the R-Forge repository
structure (i.e., \texttt{pkg} is available in the repositories) are
supported. It is possible to use 
external and internal SVN repositories side-by-side. Regardless of
origin, all packages will show up on the R packages tab.

If you decided to migrate your repository to R-Forge please go to
Section~\ref{sec:svnmigration}.

\subsection{Submit Packages to CRAN}
Packages can be \textbf{submitted easily to CRAN} through the web interface.
Note that you have to be project Administrator which indicates maintainership
of the package to achieve this. 


\section{Additional Features}

\subsection{Project Homepage}
Each project on R-Forge has its own homepage with the URL
\url{http://foo.r-forge.r-project.org/} where \texttt{foo} is the
project's unix name. Use the
pre-defined \textbf{\texttt{www}} directory in your SVN repository to
create or modify your homepage. Note that it will be checked out
daily, so please take into consideration that it will not be available
right after you commit your changes or updates. Please also note that
only html or xhtml will be allowed in the future.
% TODO: What about PHP scripting?


\subsection{SCM Options}
\label{sec:scmoptions}
Two options can be accessed through the SCM Admin tab:
\begin{enumerate}
\item{Enable anonymus access} Self-explanatory. If disabled, only
  project members can access the SVN repository and  \textbf{packages
    will not be checked out for automatic building}. The project
  homepage will still be available though. Default is enabled.
  % TODO: Fact-check this.

\item{Delivery of Commit Messages} This allows you to direct commit
  messages to an email address of your choice, e.g. the project's
  commit mailing list (see below). Default is disabled.
\end{enumerate}


\subsection{Mailing Lists}
R-Forge provides a mailing list service for all projects accessible
through the Lists tab.  

A list named \texttt{foo-commits} where \texttt{foo} is the
project's unix name is already created with each project and can be
used for distributing commit messages (see
Section~\ref{sec:scmoptions}  on how to enable
this feature). Project administrators can create and manage additional
lists for their project.  

Furthermore, \textbf{searching the lists} is provided by the Swish-e engine
and can be accessed via the List tab. Please note that private lists
are not included in the search index.  
% TODO: Reference for swish


\section{Migration from an Existing CVS/SVN Repository}
\label{sec:svnmigration}

Steps to include the complete history of your current repositroy in
your R-Forge project:
\begin{enumerate}
\item Please register a project on R-Forge first.
\item If you use CVS please convert from CVS to SVN (see e.g.,
  \url{http://www.xs4all.nl/~carlo17/svn/cvs2svn.html} on how to do
  this).
\item Then dump the whole SVN repository (using \code{svnadmin dump >
    foo.dump}).
\item Verify that the dump file can be loaded into a freshly created
  SVN repository (\code{svnadmin load newrep < foo.dump}). 
\item Send it to us: If the dump file is smaller than 10 MB you can
  send it via email to \email{R-Forge@R-project.org} otherwise you have to
  provide it somewhere for download. 
\item The last step probably includes to \code{svn move} your package
  to the /pkg directory.
\end{enumerate}


%% Acknowledgements from Rnews article
\section{Acknowledgements}

Setting up this project would not have been possible without Douglas
Bates and the University of Wisconsin as they provided us with a
server for hosting this platform. Furthermore, 
the authors would like to thank the Zentrum f\"ur Informatikdienste  
of the Wirtschaftsuniversit\"at Wien for
their support and for providing us with additional hardware as well as a
professional server infrastructure. 
Thanks to Philippe Grosjean for
providing a Mac~OS description (Section~\ref{sec:scm-macosx}).


\bibliographystyle{plainnat}
\bibliography{R-Forge_Manual}

\end{document}
