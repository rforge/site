\documentclass[a4paper]{article}

\usepackage[utf8]{inputenc}
\usepackage[authoryear,round]{natbib}
\usepackage{graphicx}
\usepackage{hyperref}
\usepackage{url}
\usepackage{a4wide}
\sloppy

\newcommand{\RFORGE}{R-Forge Administration and Development Team}

\title{R-Forge Installation and Administration Manual, ALPHA}
\author{\RFORGE}
\date{\today}

\let\code=\texttt
\let\email=\texttt
\newcommand{\pkg}[1]{{\normalfont\fontseries{b}\selectfont #1}}
\newcommand{\proglang}[1]{\textsf{#1}}
\newcommand{\class}[1]{`\code{#1}'}

\begin{document}

\pagestyle{empty}



\begin{titlepage}

  {\LARGE \raggedleft \textbf{R-Forge Installation and Administration Manual, {\large ALPHA}}}\\
%%  \vspace*{0.5cm}
  \rule{\linewidth}{1.5mm}
  \begin{flushright}
  \textbf{SVN Revision: xx, \today}\\
  \end{flushright}

  \vspace*{\fill}
%%  \vspace*{10cm} 
   
  {\Large
    \noindent {\textbf{\RFORGE{}}\\
    \rule{\linewidth}{1mm}} 
  }
  
\end{titlepage}


\newpage
\vspace*{\fill}

%% Copyright notices
%%\noindent Copyright 2006--2007 Stefan Theussl\\
\noindent Copyright 2006--2008 \RFORGE{}\\

\vspace{0.5cm}
\includegraphics{CCAL.png} 
\vspace{1cm}

\noindent The content in this manual is licensed
under a Creative Commons Attribution-Share Alike 2.0 license. 

\vspace{0.5cm}
The \RFORGE{} has chosen to apply the Creative
Commons Attribution License (CCAL) to this manual, i.e., that under the CCAL,
the authors retain ownership of the copyright 
for this manual, but the authors allow anyone to download, reuse,
reprint, modify, distribute, and/or copy the contents of this manual,
so long as the original authors and source are 
credited. Furthermore, the authors permit others to distribute
derivative works only under the same license or one compatible with
the one that governs the authors' work. 
This broad license was developed to facilitate open access
to, and free use of, original works of all types. Applying this
standard license to this work will ensure the authors' right to make
this work freely and openly available (see
\url{http://creativecommons.org/licenses/by-sa/2.0/} for details). 

\vspace{0.5cm}
The current members of the \RFORGE{} are Kurt Hornik, Martin Kober,
David Meyer, Stefan Theu\ss{}l and Achim Zeileis. To contact the
authors please write an email to \email{R-Forge@R-project.org}.

\newpage

\pagenumbering{roman}
\pagestyle{plain}
\tableofcontents

\clearpage
\pagestyle{headings}
\pagenumbering{arabic}
\setcounter{page}{1}

\section{Introduction}

Operating System Debian, reference to Rnews Article
R-Forge~User's~Manual

Subversion~\citep[SVN,
see][]{forge:Pilato+Collins-Sussman+Fitzpatrick:2004} or Concurrent
Versions System~\citep[CVS, see][]{forge:Cederqvist:2006}

\section{Installation}
\label{sec:installation}

\subsection{GForge}

\section{Components}
\label{sec:registration}

\subsection{Filesystem}
\subsection{Serverfarm---map}
\subsection{GForge}
\subsection{Postfix}
\subsection{Subversion}
\subsection{Mailman}

\section{Release and Quality Management System}
\label{sec:release_and_QM}

\subsection{Serverfarm}
\subsection{Mac OSX}
\subsection{Linux}
\subsection{Windows}


%% Acknowledgements from Rnews article
\section{Acknowledgements}

Setting up this project would not have been possible without Douglas
Bates and the University of Wisconsin as they provided us with a
server for hosting this platform. Furthermore, 
the authors would like to thank the Zentrum f\"ur Informatikdienste  
of the Wirtschaftsuniversit\"at Wien for
their support and for providing us with additional hardware as well as a
professional server infrastructure. 
Thanks to Philippe Grosjean for
providing a Mac~OS description (Section~\ref{sec:scm-macosx}).

\bibliographystyle{plainnat}
\bibliography{R-Forge_Manual}

\end{document}
