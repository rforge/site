\documentclass[12pt,a4paper]{article}
\usepackage[latin1]{inputenc}
\usepackage{graphicx}
\usepackage{latexsym}
%\usepackage{amsmath,amssymb,amsfonts}
\usepackage{ifpdf}
\usepackage{url}
\usepackage[round]{natbib}
%\usepackage{hyperref}

%% commands, environments, etc.
\let\code=\texttt
\let\email=\texttt
\newcommand{\pkg}[1]{{\normalfont\fontseries{b}\selectfont #1}}
\newcommand{\proglang}[1]{\textsf{#1}}
\newcommand{\class}[1]{`\code{#1}'}

%\renewcommand{\title}[1]{\def\@title{#1}\newcommand{\Title}{#1}}

\title{Collaborative Software Development using R-Forge}
\author{Stefan Theu\ss{}l and Achim Zeileis and Kurt Hornik}
\date{\today}

\begin{document}

\maketitle

%%\section*{Introduction}

%% Why should software developers use source code management tools?

%% Open source and its advantages
The source code of open source software should by definition be accessible to
everyone who is interested in it. Open source development is
generally claimed to produce more bug free
code and to be available faster than closed source code. Open source
developers are said to work not for monetary returns, are generally
volunteers working together on a project. These members of such a
team may come from around the world and rarely meet.

This open source
movement can
be seen as an self-organizing process which releases prototype code
frequently which is reviewed by hundreds of peers.  


R-Forge (\url{http://R-Forge.R-project.org}) is a set of tools based
on the open source software GForge (\cite{manual:gforge})---a fork of
the open source version
of SourceForge.net application that powers the world's largest open
source software repository.%% or: a system
It aims to provide a platform for collaborative development of
\proglang{R} Packages, \proglang{R} 
related software or other projects which are somehow related to our
all beloved language \proglang{R}. It provides source code management facilities
through Subversion (\cite{subversion07}) and various web-based features. 

R-Forge hosts nearly 150 Projects and has around 350 registered users at the
time of this writing.

Typically, users need to have read access to the data associated with
a certain project. Some of them (the developers) have write access to
the data. Usually there is a maintainer of the code. This person is the project
leader or has registered the project. Contributers can make use of the
provided facilities or email any changes to the code
that they developed---bug fixes, additional functionality---which the
maintainer adds to the code after verification.

An outstanding feature exclusively provided for the R community is the
possibility to have their packages daily built for Unix/Linux (the
source tarball), Windows (32~bit binary) and MacOS X (universal
binary). These packages can be 
downloaded via a download link from the website or are installable
directly in R via \code{install.packages()}.

It is shown how package developeRs are getting started with R-Forge
. Specifically this means how useRs may

\begin{itemize}
\item Register a project, 
\item Use R-Forge's version control facilities,
\item provide their packages on R-Forge,
\item host a project specific website,
\item and finally submit a package to CRAN.
\end{itemize}




\bibliography{R-Forge_abstract}

\end{document}