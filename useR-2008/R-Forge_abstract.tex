\documentclass[12pt,a4paper]{article}
\usepackage[latin1]{inputenc}
\usepackage{graphicx}
\usepackage{latexsym}
%\usepackage{amsmath,amssymb,amsfonts}
\usepackage{ifpdf}
\usepackage{url}
\usepackage[round]{natbib}
\usepackage{a4wide}
%\usepackage{hyperref}

%% commands, environments, etc.
\let\code=\texttt
\let\email=\texttt
\newcommand{\pkg}[1]{{\normalfont\fontseries{b}\selectfont #1}}
\newcommand{\proglang}[1]{\textsf{#1}}
\newcommand{\class}[1]{`\code{#1}'}

%\renewcommand{\title}[1]{\def\@title{#1}\newcommand{\Title}{#1}}

\title{Collaborative Software Development using \proglang{R}-Forge}
\author{Stefan Theu\ss{}l and Achim Zeileis and Kurt Hornik}
\date{\today}

\begin{document}

\maketitle

%%\section*{Introduction}

%% Why should software developers use source code management tools?

%% Open source and its advantages
The source code of open source software is by definition accessible to
everyone. One of the questions that may arise 
in this context is how to provide code to other developers
or even
more important how to collaborate with others to contribute to an open
source project. For this reason platforms for collaborative
development are available which
help users to find the software they like to download and allow developers to
collaborate and provide their software on a centralized storage. The
most famous example is SourceForge.net which is the world's largest
open source software development web site.

For a decade the \proglang{R} Development Core Team has been using similar
development tools like Subversion (SVN) or Concurrent Versions System
(CVS) provided by ETH Z\"urich.  
Furthermore, many \proglang{R} package developers around the world use their own
infrastructure or have their own solutions for source code
development. Therefore the \proglang{R}-project
wants to provide infrastructure not only
for the \proglang{R} Development Core Team but also for the entire \proglang{R} community
which consists of hundreds or thousands of volunteers. 

%%As
%%collaborative development is a key factor for a lot of 
%%successfull open source projects. 

%%Open source development is
%%generally claimed to produce more bug free
%%code and to be available faster than closed source code. Open source
%%developers are said to work not for monetary returns, are generally
%%volunteers working together on a project. These members of such a
%%team may come from around the world and rarely meet.
%%This open source
%%movement can
%%be seen as an self-organizing process which releases prototype code
%%frequently which is reviewed by hundreds of peers.  

\proglang{R}-Forge (\url{http://R-Forge.R-project.org}) is a set of tools based
on the open source software GForge---a fork of
the open source version
of SourceForge.net. 
It aims to provide a platform for collaborative development of
\proglang{R} packages, \proglang{R} 
related software or other projects which are somehow related to \proglang{R}.
%%our all beloved language \proglang{R}
It offers source code management facilities
through Subversion and various web-based features. 

%%R-Forge hosts nearly 150 Projects and has around 350 registered users at the
%%time of this writing.

%%Typically, users need to have read access to the data associated with
%%a certain project. Some of them (the developers) have write access to
%%the data. Usually there is a maintainer of the code. This person is the project
%%leader or has registered the project. Contributers can make use of the
%%provided facilities or email any changes to the code
%%that they developed---bug fixes, additional functionality---which the
%%maintainer adds to the code after verification.

An outstanding feature is the 
possibility to have packages built daily for  various platforms, i.e.,
Linux, MacOSX and Windows (the latter two in binary form). These
packages are accessible via download from the website and are installable
directly in \proglang{R} via \code{install.packages()} then.

In our talk we show how package developeRs are getting started with
\proglang{R}-Forge. In particular we show how people can register a project, 
use \proglang{R}-Forge's version control facilities, provide their
packages on \proglang{R}-Forge, host a project specific website, and
finally submit a package to CRAN.
\end{itemize}

\end{document}