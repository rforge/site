\documentclass[12pt,a4paper]{article}
\usepackage[latin1]{inputenc}
\usepackage{graphicx}
\usepackage{latexsym}
%\usepackage{amsmath,amssymb,amsfonts}
\usepackage{ifpdf}
\usepackage{url}
\usepackage[round]{natbib}
\usepackage{a4wide}
%\usepackage{hyperref}

%% commands, environments, etc.
\let\code=\texttt
\let\email=\texttt
\newcommand{\pkg}[1]{{\normalfont\fontseries{b}\selectfont #1}}
\newcommand{\proglang}[1]{\textsf{#1}}
\newcommand{\class}[1]{`\code{#1}'}

%\renewcommand{\title}[1]{\def\@title{#1}\newcommand{\Title}{#1}}

\title{Collaborative Software Development Using \proglang{R}-Forge}
\author{Stefan Theu\ss{}l and Achim Zeileis and Kurt Hornik}
\date{\today}

\begin{document}

\maketitle

%%\section*{Introduction}

%% Why should software developers use source code management tools?

%% Open source and its advantages
A key factor in open source software development is the rapid creation
of solutions within an open, collaborative environment. The open
source model had its major breakthrough with the increasing
usage of the internet. Online communities successfully combined
not only their programming effort but also their knowledge, work
and even their social life.
 
The consequence was an increasing demand for centralized resources e.g.,
to manage projects or source code. The most famous of such
platforms---the world's largest open source
software development web site---is SourceForge.net.

For a decade, the \proglang{R} Development Core Team as well as many
\proglang{R} package developeRs have been using
development tools like Subversion (SVN) or Concurrent Versions System
(CVS) for managing their source
code. A central repository is hosted by ETH Z\"urich mainly for
managing the development of the base \proglang{R} system. Now, the
\proglang{R}-project wants to provide infrastructure for the entire
\proglang{R} community.

%%One of the questions that may arise 
%%in this context is how to provide code to other developers
%%or even
%%more important how to collaborate with others to contribute to an open
%%source project. centralized resource for managing projects and
%%source code

%%As
%%collaborative development is a key factor for a lot of 
%%successfully open source projects. 

%%Open source development is
%%generally claimed to produce more bug free
%%code and to be available faster than closed source code. Open source
%%developers are said to work not for monetary returns, are generally
%%volunteers working together on a project. These members of such a
%%team may come from around the world and rarely meet.
%%This open source
%%movement can
%%be seen as an self-organizing process which releases prototype code
%%frequently which is reviewed by hundreds of peers.  

\proglang{R}-Forge~(\url{http://R-Forge.R-project.org}) is a set of tools based
on the open source software GForge---a fork of
the open source version of SourceForge.net.
It aims to provide a platform for collaborative development of
\proglang{R} packages, \proglang{R} 
related software or other projects which are somehow related to \proglang{R}.
%%our all beloved language \proglang{R}
It offers source code management facilities through SVN and
a wide variety of web-based services.

%%R-Forge hosts nearly 150 Projects and has around 350 registered users at the
%%time of this writing.

%%Typically, users need to have read access to the data associated with
%%a certain project. Some of them (the developers) have write access to
%%the data. Usually there is a maintainer of the code. This person is the project
%%leader or has registered the project. Contributers can make use of the
%%provided facilities or email any changes to the code
%%that they developed---bug fixes, additional functionality---which the
%%maintainer adds to the code after verification.

Furthermore, packages hosted on \proglang{R}-Forge are built daily
for  various operating systems, i.e., Linux, MacOSX and Windows. These
package builds are downloadable from the 
project's website on \proglang{R}-Forge as well as installable 
directly in \proglang{R} via \code{install.packages()}.

In our talk we show how package developeRs can get started with
\proglang{R}-Forge. In particular we show how people can register a project, 
use \proglang{R}-Forge's source code management facilities, provide their
packages with \proglang{R}-Forge, host a project specific website, and
finally submit a package to CRAN.

\end{document}
