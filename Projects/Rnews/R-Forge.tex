\title{Collaborative Software Development Using R-Forge}
\author{Stefan Theu\ss{}l and Achim Zeileis}

\maketitle

%% Story of the article:
%% (1) Open source software (OSS) development - history, importance  _/
%%     - Apache, Linux famous examples ?TODO
%%     - Scripting languages, PHP, Perl and of course R ?TODO
%%     - software repositories: CPAN, CTAN -> CRAN ?TODO 
%% (2) collaboration and project management  _/
%%     - collaboration sites: SourceForge.net _/
%% (3) why is it important to have a central repository TODO
%%     - The Cathedral and the Bazaar TODO
%%     - large community -> better TODO
%% (4) - R-Forge  _/
%%     - explaining the basic tools  _/
%%     - R specific features  _/
%%       Package building/checking facilities
%%     - quality management through R check and tracker _/
%%     - other stuff: forums  _/

%% (5) How to forge packages on R-Forge (A day in a life of a new R
%%     package developer).
%%     - registering as a user  _/
%%     - registering a project (automatically creates SVN, a
%%       mailinglist -> managable through web interface -> click to send
%%       commit statements to this list).  _/
%%     - the svn base tree (picture of an axample tree with more
%%       packages) TODO


\section*{Introduction}

%% from useR abstract 
%% <useR>
%% Why should software developers use source code management tools?

A key factor in open source software (OSS) development is the rapid creation
of solutions within an open, collaborative environment. The open
source model had its major breakthrough with the increasing
usage of the internet. Online communities successfully combined
not only their programming effort but also their knowledge, work
and even their social life. OSS is by definition free and
the source code is available
claimed to be  because of the
``release early, release often'' paradigm introduced by Linus
Torvalds.


\textbf{TODO: } I don't know what to do with the following paragraph
%% FIXME:
Much early work on OSS development was aimed at raising awareness  OSS
work produced by a community of developers ``The Cathedral and the
Bazaar'' by \cite{forge:Raymond:1999}.
Advantages:
decentralized self-organizing process, rapid code evolution, massive
peer code review, rapid release of prototype code
Developers of larger projects like the language \R{} use source code
management (SCM) facilities to follow this paradigm.
The \R{} Development Core Team have been using
development tools like Subversion \citep[SVN,
see][]{forge:Pilato+Collins-Sussman+Fitzpatrick:2004} or  
Concurrent Versions System \citep[CVS, see][]{forge:Cederqvist:2006}
for managing their source code for a decade now.  
A central repository is hosted by ETH Z\"urich mainly for
managing the development of the base \R{} system. The
\R{}-project is now ready to provide similar infrastructure
for the entire \R{} community.
R-Forge (\url{http://R-Forge.R-project.org}) provides a set of tools
for source code management and various web-based
features. It aims to provide a platform for collaborative development of
\R{} packages, \R{}-related software or further projects. But it does not
matter if individuals work on their own or a number of developers
collaborate in a larger project, on R-Forge all work is organized in
the same entity namely a ``Project''.

R-Forge is closely related to the most famous of such platforms---the
world's largest OSS development website---namely
\url{http://SourceForge.net}. In addition to this platform it provides
tools especially for the \R{} community.

%% Article Outline
The present article is organized as follows. First, we present the core
features that R-Forge is offering to the \R{} community. Second we
show how package developers can get started with 
\R{}-Forge. In particular we answer some questions like: How people can
register a project, use \R{}-Forge's source code management facilities, provide their
packages with \R{}-Forge, host a project specific website, and
finally how to submit a package to CRAN
(\url{http:\\CRAN.R-project.org}).
Eventually we summarize recent developments and give a brief outlook
to future work.

%% What is R-Forge offering to you?
%% Alternative title: Core Features of R-Forge
\section{R-Forge}
%% what is R-Forge?
R-Forge offers developers a central platform for the development of \R{} packages, \R{}
related software and further projects. On R-Forge developers organize their work
in so called projects. Every project has various tools and web-based features
for software development, communcation between developers themselves
as well as their
user base and other services enabled.

R-Forge is based on the OSS GForge \citep{forge:copeland_et_al:2006} which is a
fork of the 2.61 SourceForge code maintained by one of the authors of
the original SourceForge code Tim Perdue. GForge has been modified to
provide additional features for the \R{} community namely a
CRAN-style repository for hosting development releases of \R{}
packages as well as a quality management system similar to that of
CRAN.
%%This and the especially of interest for the \R{} community
%% brief overview of features possibly most interesting for readers
Packages hosted on \R{}-Forge are built daily
for various operating systems, namely Linux, MacOSX and Windows. These
package builds are downloadable from the website of the project on
\R{}-Forge or are installable directly in \R{} via
\code{install.packages("packagename",
  repos="http://R-Forge.R-project.org")}.

%%Tabs can be distinguished in main
%%tabs (currently Home, My Page and Project Tree) and project specific
%%tabs which are only visible when you access a project.

%% Additional Information to GForge. Since the codebase of 
%% SourceForge has not been released over a certain time the GForge
%% project was formed. 

%% Why should software developers use source code management tools?
%% FIXME: How to formulate better
%% Version control - probably the most important feature

%%Probably the most important features correspond to 3 project tabs are
%%``Summary'' which offers an overview of the whole project, ``SCM''
%%describing how people can access the source code repository and ``R
%%Packages'' containing a list of packages available in this Project.

%% eventually include history of the development of R-Forge
R-Forge hosted nearly 200 Projects and had around 450
registered users in its development stage. This and the steadily growing list of
feature requests show that there is a high demand for centralized source code
management tools and for releasing prototype code frequently among the
\R{} community.

In the next section we explain the features which are important for
collaborative development and for releasing prototype or development
packages on R-Forge.
%% TABS and projects
All features mentioned in the following sections are accessible via so called
``tabs''. E.g., user accounts can be managed in the ``My page'' tab or
a list of available projects can be displayed using the ``Project Tree'' tab. 


\subsection{Source Code Management}

When carrying out a software projects source files change over time,
new files get created and old files deleted. Moreover several authors
work on several computers on the same and/or different files. Often it
is sufficient to have a shared storage for the source code but if the
project reaches a certain size it becomes a tedious task to overlook
every file change. Especially if files get overwritten or deleted it
is often not easy to restore them from a backup. Source
code management tools like SVN help in organizing the process of
creating a software. A central repository ensures that the developer
has always access to the current version of the project source
code. Any of the authorized collaborators can ``checkout'' or
``update'' the project
file structure, make the necessary changes or additions, delete
files from the project and finally ``commit'' his changes or addions
to the repository. More than
that, SVN keeps track of the complete history of the project file
structure. At any point in the development stage it is possible to go
back to any stage in the history as well as to inspect and restore old
files. This is called version control as every stage automatically is
assigned a unique version number which increases over time. 
All of this enables a group of developers to work on a given project
simultaneously without losing time thinking of backups or thinking of
details like throwing away obsolete code as there is always the possibility to 
go back to a specific version of the whole software repository.

%% Summarize the SCM feature
On R-Forge such a version controlled repository is automatically
created for each project. The project members just have to install
the client of their choice (e.g., TortoiseSVN on Windows of svnX on
MacOSX), check out the repository and can start working using all the
features provided by SVN. In addition to the inherent backup of every
version mentioned above a backup of the whole repository and thus of
the whole history of the project is made daily. This backup is necessary if
the repository itself gets corrupted. 

%% except two predefined directories (\code{pkg} and \code{www})

%% Rights management on R-Forge
Typically, all users have read access and some of them (the
developers) have write access to the data associated with
a certain project. On R-Forge a rights management system assures that
registered users can be granted one of several so called roles. 
%% Explaining rights management
E.g., the \textit{Administrator} has all rights including the right to
add new users to the project or release packages directly to CRAN
using the ``R Packages'' tab. He or she is usually the package
maintainer, the project leader or has registered the project.
Other members of a project typically have either the role \textit{Senior 
Developer} or \textit{Junior Developers}. These two roles are
permitted to commit to the project 
SVN repository and examine the log files in the ``R Packages'' tab.
When we speak of developers in subsequent sections we refer to project
members having the rights of at least of a \textit{Junior Developer}.


\subsection{Release and Quality Management of R Packages}

%% Quality Management and bug/support/feature tracker
When realising early version of a software project they are typically
prototypes and therefore are 
not completely bug free. R-Forge can assist the developer with two
tools to improve the quality of their products. First it offers a
quality management system similar to 
that of CRAN. Packages on R-Forge are checked in a
standardized way based on \code{R CMD check} on different
platforms. The resulting log files can be examined by the project
developers so that they can make the necessary changes in order to get
the package working properly on each platform.

Second, a bug tracking systems allows users  to notify
the authors of the package about problems they encountered. 
Peer code review is one of the major advantages of OSS and lead to an
overall improvent of the quality of the projects---Given
enough eyeballs, all bugs are shallow \citep{forge:Raymond:1999}.

\subsection{Additional Features}

Additionally, \textbf{R-Forge} provides other tools especially for
communication between project members and their user base. Further
tools help developers of larger projects to coordinate their work.
What follows is a brief summary of these tools.
\begin{itemize}
\item Project websites are a way for developers to present their work on a
subdomain of \url{http://r-forge.r-project.org}. It is also allowed to
have a link on the project summary page to another website. 
%% checked out hourly from ``www'' directory of the
%%project's repository.
\item Mailing Lists are nowadays valuable for every kind of
  project. By default one is automatically created when setting up a
  project. Additional mailing lists can be set up as well. 
\item The project tree offers the assignment of a project to one or more
  topics (e.g.: biostatistics, finance, regression analysis,
  \ldots). This enables other people to quickly find what they are
  looking for. In the default setting the project tree lists
  alphabetically all projects including a short description similar to CRAN.
\item Forums can be set up separately by the project
  administrators.%% are places to discuss certain topics 
\item News can be put on the project summary page as well as on the
  front page. The latter needs an approval by one of the R-Forge
  administrators). It is possible to download them as RSS feeds.
\end{itemize}

\section{How to get started}
%%A small guide through R-Forge---TODO: detailed description 
In this section we show how new users can get started with
R-Forge. For a more detailed guide to R-Forge and additional
information we refer to the user's manual
\citep{forge:theussl:2008}.

If you type the URL \url{http://R-Forge.R-project.org} into your
browser you get to the main page. Here you have the following
opportunity to
\begin{itemize}
\item login,
\item register a user or a project,
\item download the documentation,
\item examine the latest news (also available as RSS feed),
\item enter a specific project website either by search for available
  projects using the search field on the top middle of the page, going
  through an alphabetically listing seen in the
  ``Project Tree'' tab or by clicking on one of the projects listed on
  the right of the front page,
\item and examine R-Forge specific statistics (how many projects/users
  are currently registered on R-Forge, project activity percentages,
  \ldots{}).
\end{itemize}

There is another tab on the front page named ``My Page'', where you
can configure your account. But before you can use it, you have to
register yourself as a user.

\subsection{Registering as a New User}

What has to be done in the first place to get started with a project
on R-Forge is to register yourself as a user on R-Forge. There is a link on
the main web site called ``New Account'' on the top right of the page
which leads to the registration form.

After submitting the form (filled with your name, email address, and
some other information) an email gets sent to the given mail
address containing a link for activating your account. After you
activated your account you are able to join an existing project or
create a new one.
The latter is probably more of interest if you decide to migrate an
existing package to R-Forge. Therefore we explain the registration 
of a project in more details. For joining an existing project we refer
to the user's manual.

\subsection{Registering a Project}

To achieve this either go to the main page and click on ``Register
Your Project'' or go to the ``My Page'' tab and click on ``Register
Project''. You are presented with a new form which you have to fill
out in order to register your project. Some remarks: Only the Project
Public Description will get visible on the project's homepage, whereas
the Project Purpose And Summarization only is an information for
us---the R-Forge administrators. The Project Unix Name is a name which
uniquely defines your project (e.g., it can be the name of your
package). There are several restrictions to it which are explained on this
page. E.g., according to the UNIX
file system conventions these names have to be in lower case (and will
be converted to it automatically in case you typed in upper case
characters).

After filling out the form and submitting it, the R-Forge
administrators decides whether this project gets approved or not. This
is only to prevent fake projects to get registered. After approval a
confirmation email is sent to the project administrator and the svn
repositories are created (note: this is done every hour on a fixed
time, so it is possible that you have to wait until you can check out
your repository).

To manage your project the administrator has several
possibilities. But first I want to explain what is important to know
about the project main page.

Every project has some sort of a own web area on R-Forge. Actually
there are two of them. A standardized one (which is the same for every
project on R-Forge) and a personalized one. The latter is accessible
through <Project Unix Name>.R-Forge.R-project.org and is managed via
the SVN repository (www directory). For useRs and developeRs the first
one may be more important and is therefore explained here in more
details.

On the Project Summary Page (typically the initial one when entering a
project) you see the following

\begin{itemize}
\item some details about the project (a short description,
  administrators, developers, \ldots{})

\item a download link which leads directly to the available packages
  of the project.
\item a link leading to the (personal) project homepage

\item links to activated features of the project (like tracker,
  Forums, Mailing Lists, \ldots{})

\item you'll find the latest news announcements (if made any)

\end{itemize}

In the project area each available feature is organized in a separate
tab. For example if you like to ask a question or discuss something
with the developers of the project you may click on the ``Forums'' tab
and enter the corresponding forum. The same is possible with mailing
lists. Actually, there is setup an initial mailing list called
<name>-commits which can be used to get commit message to the SVN
repository delivered (this has to be activated separately). New
mailing lists can be easily set up.

One of the most important tab is the R Packages tab. But before that
let us explain a bit the build process of R-Forge.

Every night a script parses the pkgs directories and exports the
packages found to a specific place. From this place the build machines
(currently a linux, mac und windows machine) sync these packages and
start to build them. Meanwhile the links to the source and the
binaries in the R packages tab get updated. After building is finished
the builds get synced back to R-Forge. They are available for download
then. Furthermore logs have been produced and are also available for
examination.

Another feature is the quality check we run every day. Check results
are also provided on the R tab.

%% some remarks to the build process

\section*{Recent and future developments}
In this section we summarize the developments which took place between
the first release basically consisting of the features of GForge and
the current release Version. Then we briefly give an outlook to future
developments.

Recently added features and major changes include:
\begin{itemize}
\item Enhanced structure in the SVN repository allows multiple
  packages in one project. Each package is a subdirectory of the
  \code{/pkg} directory.
\item An additional check box on the news submit page helps to
  separate front page news from project-only news. Nevertheless,
  whenever a news item is declared to be put on the front page,
  R-Forge administrators need to approve them (default project-only
  submission).
\item A new button in the SCM Admin tab allows for delivery of SVN
  commit messages (default: off).  This
  list is used for circulating SVN commit messages to registered
  users (has to be turned on separately).
\item In newly registered projects by default only the SCM, the R
  packages tab and mailing lists are turned on. Forums, News and
  Tracker can be turned on separately using ``Edit Public Info'' in
  the Admin tab of the project. This is aimed to help new users to get
  in touch with R-Forge and to let users decide which features they
  want to use.
\item We moved to a platform independent package building and quality
  management code combined in one \R{} package.
\end{itemize}

Currently on the wishlist and under development:
\begin{itemize}
\item Wiki,
\item Task management,
\item Restructured tracker more compatible to \R{}. E.g., \R{} version
  used which generated the bug.  
\end{itemize}

%% - 

\section{Acknowledgements}

Setting up this project would not have been possible without Douglas
Bates and the University of Wisconsin as they provided us with a
server for hosting this platform. Furthermore, 
the authors would like to thank the Computer Center 
of the Wirtschaftsuniversit\"at Wien for
their support and for providing us with additional hardware as well as a
professional server infrastructure.

\bibliography{R-Forge}