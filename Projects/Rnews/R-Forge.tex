\title{Collaborative Software Development Using R-Forge}
\author{Stefan Theu\ss{}l and Achim Zeileis}

\maketitle

%% Story of the article:
%% (1) Open source software (OSS) development - history, importance  _/
%%     - Apache, Linux famous examples ?TODO
%%     - Scripting languages, PHP, Perl and of course R ?TODO
%%     - software repositories: CPAN, CTAN -> CRAN ?TODO 
%% (2) collaboration and project management  _/
%%     - collaboration sites: SourceForge.net _/
%% (3) why is it important to have a central repository TODO
%%     - The Cathedral and the Bazaar TODO
%%     - large community -> better TODO
%% (4) - R-Forge  _/
%%     - explaining the basic tools  _/
%%     - R specific features  _/
%%       Package building/checking facilities
%%     - quality management through R check and tracker _/
%%     - other stuff: forums  _/

%% (5) How to forge packages on R-Forge (A day in a life of a new R
%%     package developer).
%%     - registering as a user  _/
%%     - registering a project (automatically creates SVN, a
%%       mailinglist -> managable through web interface -> click to send
%%       commit statements to this list).  _/
%%     - the svn base tree (picture of an axample tree with more
%%       packages) TODO

%% new command for R-Forge tabs
\newcommand{\tab}[1]{{\normalfont\textit{#1}}}

\section*{Introduction}

%% from useR abstract 
%% <useR>
%% Why should software developers use source code management tools?

A key factor in open source software (OSS) development is the rapid creation
of solutions within an open, collaborative environment. The open
source model had its major breakthrough with the increasing
usage of the internet. Online communities successfully combined
not only their programming effort but also their knowledge, work
and even their social life. OSS is by definition free and
the source code is available to everyone.
claimed to be  because of the
``release early, release often'' paradigm introduced by Linus
Torvalds.
%% Peer review of open source software


\textbf{TODO: } I don't know what to do with the following paragraph
%% FIXME:
Much early work on OSS development was aimed at raising awareness  OSS
work produced by a community of developers ``The Cathedral and the
Bazaar'' by \cite{forge:Raymond:1999}.
Advantages:
decentralized self-organizing process, rapid code evolution, massive
peer code review, rapid release of prototype code
Developers of larger projects like the language \R{} use source code
management (SCM) facilities to follow this paradigm.

The \R{} Development Core Team have been using
development tools like Subversion \citep[SVN,
see][]{forge:Pilato+Collins-Sussman+Fitzpatrick:2004} or  
Concurrent Versions System \citep[CVS, see][]{forge:Cederqvist:2006}
for managing their source code for a decade now.  
A central repository is hosted by ETH Z\"urich mainly for
managing the development of the base \R{} system. 

Many package developers use similar infrastructure to manage their
source code, e.g., in private SVN repositories, using web tools like
Google Code (\url{http://code.google.com}) or simply by using shared
storage. Additionally, when examining the \file{DESCRIPTION} files of
packages on CRAN we see that around $47\%$ of them have multiple
authors (as of August 10, 2008). All of the above show that SCM
facilities are already in use in the \R{} community and that a central
repository would help people to get started with these tools.

The \R{}-project is now ready to provide infrastructure
for the entire \R{} community. 
R-Forge (\url{http://R-Forge.R-project.org}) provides a set of tools
for source code management and various web-based
features. It aims to provide a platform for collaborative development of
\R{} packages, \R{}-related software or further projects. But it does not
matter if individuals work on their own or a number of developers
collaborate in a larger project, on R-Forge all work is organized in
the same entity namely a ``Project''.

R-Forge is closely related to the most famous of such platforms---the
world's largest OSS development website---namely
\url{http://SourceForge.net}. The major strength of \R{}-Forge in
comparison to SourceForge is the focus on services especially aimed at
the \R{} community.

%% Article Outline
The present article is organized as follows. First, we present the core
features that \R{}-Forge offers to the \R{} community. Second, we
give a hands on tutorial on how package developers can get started with 
\R{}-Forge. In particular we show how people register new projects,
use \R{}-Forge's source code management facilities, provide their 
packages with \R{}-Forge, host a project specific website, and
finally submit a package to the Comprehensive R Archive Network (CRAN,
\url{http://CRAN.R-project.org}).
Eventually we summarize recent developments and give a brief outlook
to future work.

%% What is R-Forge offering to you?
%% Alternative title: Core Features of R-Forge
\section{R-Forge}
%% what is R-Forge?
R-Forge offers a central platform for the development of \R{} packages, \R{}
related software and further projects. 

R-Forge is based on the OSS GForge \citep{forge:copeland_et_al:2006} which is a
fork of the 2.61 SourceForge code maintained by one of the authors of
the original SourceForge code Tim Perdue. GForge has been modified to
provide additional features for the \R{} community namely a
CRAN-style repository for hosting development releases of \R{}
packages as well as a quality management system similar to that of
CRAN.
%%This and the especially of interest for the \R{} community
%% brief overview of features possibly most interesting for readers
Packages hosted on \R{}-Forge are built daily
for various operating systems, namely Linux, Mac OSX and Windows. These
package builds are downloadable from the website of the project on
\R{}-Forge or are installable directly in \R{} via
\code{install.packages("foo",
  repos="http://R-Forge.R-project.org")}.

%%Tabs can be distinguished in main
%%tabs (currently Home, My Page and Project Tree) and project specific
%%tabs which are only visible when you access a project.

%% Additional Information to GForge. Since the codebase of 
%% SourceForge has not been released over a certain time the GForge
%% project was formed. 

%% Why should software developers use source code management tools?
%% FIXME: How to formulate better
%% Version control - probably the most important feature

%%Probably the most important features correspond to 3 project tabs are
%%\tab{Summary} which offers an overview of the whole project, \tab{SCM}
%%describing how people can access the source code repository and \tab{R
%%Packages} containing a list of packages available in this Project.

%% eventually include history of the development of R-Forge

On R-Forge developers organize their work
in so-called projects. Every project has various tools and web-based features
for software development, communcation and other services enabled. All
features mentioned in the 
following sections are accessible via so called 
``tabs''. E.g., user accounts can be managed in the \tab{My~Page} tab or
a list of available projects can be displayed using the
\tab{Project~Tree} tab.

Since starting the platform in early 2007 more
and more interested users registered their projects on R-Forge. Now
after a one and a half years being in a development and testing stage nearly
240 Projects and around 600  
users are registered on \R{}-Forge. This and the steadily growing list of
feature requests show that there is a high demand for centralized source code
management tools and for releasing prototype code frequently among the
\R{} community.

In the next section we explain the features which are important for
collaborative development and for releasing prototype or development
packages on R-Forge.
%% TABS and projects

\subsection{Source Code Management}

When carrying out software projects source files change over time,
new files get created and old files deleted. Typically, several authors
work on several computers on the same and/or different files. Often it
is sufficient to have a shared storage for the source code but if the
project reaches a certain size it becomes a tedious task to keep track
of every file change. Especially if files get overwritten or deleted it
is often not easy to restore them from a backup. Source
code management tools like SVN help in organizing the process of
creating a software. A central repository ensures that the developer
has always access to the current version of the project's source
code. Any of the authorized collaborators can ``checkout'' or
``update'' the project
file structure, make the necessary changes or additions, delete
files from the project and finally ``commit'' his or her changes or additions
to the repository. More than
that, SVN keeps track of the complete history of the project file
structure. At any point in the development stage it is possible to go
back to any stage in the history as well as to inspect and restore old
files. This is called version control as every stage automatically is
assigned a unique version number which increases over time. 
All of this enables a group of developers to work on a given project
simultaneously without losing time thinking of backups or thinking of
details like throwing away obsolete code as there is always the possibility to 
go back to a specific version of the whole software repository.

%% Summarize the SCM feature
On R-Forge such a version controlled repository is automatically
created for each project. The project members just have to install
the client of their choice (e.g., Tortoise SVN on Windows or svnX on
Mac OSX), check out the repository and then are ready to start working
using all the features provided by SVN. In addition to the inherent
backup of every version mentioned above a backup of the whole
repository and thus of the whole history of the project is made daily. 

%This backup is necessary if
%the repository itself gets corrupted. 

%% except two predefined directories (\code{pkg} and \code{www})

%% Rights management on R-Forge
Typically, all users have read access and some of them (the
developers) have write access to the data associated with
a certain project. On R-Forge a rights management system assures that
registered users can be granted one of several roles. 
%% Explaining rights management
E.g., the \textit{Administrator} has all rights including the right to
add new users to the project or release packages directly to CRAN. He
or she is usually the package 
maintainer, the project leader or has registered the project originally.
Other members of a project typically have either the role \textit{Senior 
Developer} or \textit{Junior Developers}. These two roles are
permitted to commit to the project 
SVN repository and examine the log files in the \tab{R Packages} tab.
When we speak of developers in subsequent sections we refer to project
members having the rights of at least of a \textit{Junior Developer}.


\subsection{Release and Quality Management of R Packages}
\label{sec:release_and_quality_management}
%% Quality Management and bug/support/feature tracker
When releasing early version of a software project they are typically
prototypes and therefore are 
not completely bug free. \R{}-Forge assists the developers with two
tools to improve the quality of their products. First, it offers a
quality management system similar to 
that of CRAN. Packages on \R{}-Forge are checked in a
standardized way on different platforms based on
\code{R~CMD~check}. The resulting log files can be examined by the
project developers so that given the check summary they can improve
the package to pass all tests on \R{}-Forge and subsequently on CRAN.

Second, bug tracking systems allow users to notify
package authors about problems they encountered. In the spirit of
OSS---given enough eyeballs, all bugs are shallow
\citep{forge:Raymond:1999}---peer code review leads to an 
overall improvement of the quality of software projects.

\subsection{Additional Features}

A number of other tools, especially of interest for larger
projects, help developers to coordinate their work and to communicate
with their user base. These tools include:

\begin{itemize}
\item Project websites: Developers may present their work
  on a subdomain of R-Forge, e.g.,
  \url{http://foo.R-forge.R-project.org}), or via a link to an
  external website.
%% checked out hourly from ``www'' directory of the
%%project's repository.
\item Mailing lists: By default one is automatically created when setting up a
  project. Additional mailing lists can be set up easily. 
\item Project categorization: Different
  topics (e.g., biostatistics, finance, regression analysis,
  \ldots) enable users to quickly find what they are
  looking for by browsing these categories in the so-called
  \tab{Project Tree} tab. By default an alphabetical listing of all
  projects each including a short description similar to CRAN is
  presented.
\item News: Announcements and other information related to a project
  can be put on the project summary page as well as on the 
  home page of \R{}-Forge. The latter needs approval by one of the R-Forge
  administrators. All items are available as RSS feeds.
\item Forums: Discussion forums have to be set up separately by the
  project administrators.%% are places to discuss certain topics 
\end{itemize}

\section{How to get started}
%%A small guide through R-Forge---TODO: detailed description 
This section is intended to be a hands on tutorial for new users so
that they can easily get started with
\R{}-Forge. For a more detailed guide to \R{}-Forge and additional
information we refer to the user's manual
\citep{forge:usermanual:2008}.

After accessing the URL \url{http://R-Forge.R-project.org} the home
page is presented to the user. Here one can
\begin{itemize}
\item login,
\item register a user or a project,
\item download the documentation,
\item examine the latest news,
\item go to a specific project website either by searching for available
  projects (top middle of the page), by clicking on one of the projects
  listed on the right, or by going through an alphabetical listing in
  the \tab{Project Tree} tab. 
\item examine R-Forge specific statistics (how many projects/users
  are currently registered on R-Forge, project activity percentages,
  \ldots{}).
\end{itemize}

Another tab on the front page named \tab{My Page} leads the user to
their personal page. Here, registered users can configure their
account among other things.

How upcoming R-Forge developers can register as new users is
explained in the next section.

\subsection{Registering as a New User}

Before using certain features or starting projects on R-Forge one has
to register as a site user. A link on the main web site called
\tab{New Account} on the top right of the home page leads to the
corresponding registration form.

After submitting the completed form an email is sent to the given mail
address containing a link for activating the account. Having completed
the registration the user is allowed to participate in all \R{}-Forge
has to offer, like joining an existing project or creating a new one,
e.g. when a new package is developed or existing packages are to be
migrated to R-Forge. In the following section we explain 
the registration of a project in more details. For joining an existing
project we refer to the user's manual.

\subsection{Registering a Project}

There are two possibilities to register a project: Clicking on
\tab{Register Your Project} on the home page or by going to the \tab{My
Page} tab and clicking on \tab{Register Project}. Both links lead to a
form which has to be filled out in order to finish the registration
process. Note that only the text field ``Project Public Description''
is relevant on \R{}-Forge, whereas ``Project Purpose And
Summarization'' is only an information for the R-Forge
administrators. ``Project Unix Name'' refers to the name which
uniquely determines the project (e.g., the name of a
package). Restrictions according to the Unix file system convention 
force Unix names to be in lower case (and will
be converted automatically in case they are typed in upper
case).

After submitting the completed form and the approval of the R-Forge
administrators a confirmation email is sent to the registrant, who
automatically becomes the project administrator. To present the new
project to a broader community the name of the project additionally is
promoted on the home page under ``Recently Registered
Projects''. Since then the standardized web area of the project is
available on \R{}-Forge. 


Entering a project typically leads to the \tab{Summary}
page which allows the user to

\begin{itemize}
\item examine the details about the project including a short
  description and a listing of the administrators and developers,
\item follow the download link leading directly to the available packages
  of the project (i.e., the \tab{R Packages} tab),
\item follow a link leading to the (personal) project homepage,
\item examine the latest news announcements (if available),
\item go to other sections of the project like
  \tab{Forums}, \tab{Tracker},
   \tab{Lists}, \tab{R Packages}, \ldots{}.
\end{itemize}

As stated in the listing above a second web area, accessible 
through \url{http://foo.R-Forge.R-project.org} where \code{foo}
corresponds to the unix name of the project, is automatically
created. This project website is managed via the \code{www} directory
in the project's SVN repository. Plain html 
is supported and the homepage gets updated every hour.

Last but not least it should be noted that the SVN repository as well
as the default mailing list are available within an hour after
approval. Is the SVN repository set up and running the next step would
be to migrate packages to R-Forge. This is explained in the next
section.

\subsection{SCM and R Packages}

The \tab{SCM} tab of a project explains how the corresponding SVN
repository can be checked out to your PC. Typically, this is done either
anonymously without write permission (if enabled by the
administrator---the default)
or with full access to the repository as a developer. 

To make use of the package building and checking feature mentioned
above the package source code has to be put into the \code{pkg}
directory or, alternatively, a 
subdirectory of \code{pkg}. A major advantage of this structure is to
have more than one package in a single project. Then the corresponding
sources need to be in different subdirectories of \code{pkg}. E.g., if
a project consists of the packages \code{foo} and \code{bar} then the
source code is located in \code{pkg/foo} and \code{pkg/bar},
respectively.


After committing the package/s to the server the tarball containing
the package sources as well as the package binaries will be available
within a day in the \tab{R Packages} tab for download. Furthermore,
logs of the build and check process on different platforms 
can be examined by the developers. The packages
are installable directly in \R{} using \code{install.packages("foo",
  repos="http://R-Forge.R-project.org")}. 

To release a package to CRAN the project administrator clicks on
\tab{Submit this package to CRAN} in the \tab{R Packages} tab. After
confirmation that a message will be sent to \email{CRAN@R-project.org}
the package is automatically copied to
\url{ftp://cran.r-project.org/incoming}.

It is also possible that developers migrate an existing repository
including the complete history to \R{}-Forge. For this we refer to the
user's manual.

\section*{Recent and future developments}

In this section we summarize the developments which took place between
the first release basically consisting of the features of GForge and
the current release Version. Then we briefly give an outlook to future
developments.

Recently added features and major changes include

\begin{itemize}
\item an enhanced structure in the SVN repository allowing multiple
  packages in a single project. Each package is located in a
  subdirectory of the \code{pkg} directory.
\item an additional check box on the news submit page helping the
  \R{}-Forge administrators to separate front page news from
  project-only news. Nevertheless, news items declared to be put on
  the front page need approval (default: project-only submission).
\item a new button in the SCM Admin tab allowing for circulation of SVN
  commit messages. The mailing list mentioned in the text field is
  used for delivering the SVN commit messages (default: off).
\item new defaults for freshly registered projects: only the tabs
  \tab{Lists}, \tab{SCM} and\tab{R~packages} are turned on
  initially. \tab{Forums}, \tab{Tracker} and \tab{News}
  can be turned on separately using \tab{Edit Public Info} in
  the \tab{Admin} tab of the project. This is aimed to not overwhelm
  new users with a flood of features. Experienced users can decide
  which features they want and activate them.
\item an R{} package containing platform independent package building and quality
  management code used on the \R{}-Forge servers.
\item Mailing list search facilities provided by the Swish-e engine
  and can be accessed via the \tab{List} tab (private lists
  are not included in the search index).

\end{itemize}

Eventually, we briefly summarize the features which are currently on
the wishlist or are under development and likely to be included in the
near future. Among others these are

\begin{itemize}
\item a Wiki,
\item task management facilities,
\item a re-organized tracker more compatible to \R{} package development. 
\end{itemize}

For suggestions, wishes, help requests and other questions regarding
\R{}-Forge please write an email \email{R-Forge@R-project.org}.

\section{Acknowledgements}

Setting up this project had not been possible without Douglas
Bates and the University of Wisconsin as they provided us with a
server for hosting this platform. Furthermore, 
the authors would like to thank the Computer Center 
of the Wirtschaftsuniversit\"at Wien for
their support and for providing us with additional hardware as well as a
professional server infrastructure.

\bibliography{R-Forge}