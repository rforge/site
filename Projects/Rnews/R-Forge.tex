\title{Collaborative Software Development Using R-Forge}
\author{Stefan Theu\ss{}l and Achim Zeileis}

\maketitle

%% Story of the article:
%% (1) Open source software (OSS) development - history, importance  _/
%%     - Apache, Linux famous examples ?TODO
%%     - Scripting languages, PHP, Perl and of course R ?TODO
%%     - software repositories: CPAN, CTAN -> CRAN ?TODO 
%% (2) collaboration and project management  _/
%%     - collaboration sites: SourceForge.net _/
%% (3) why is it important to have a central repository TODO
%%     - The Cathedral and the Bazaar TODO
%%     - large community -> better TODO
%% (4) - R-Forge  _/
%%     - explaining the basic tools  _/
%%     - R specific features  _/
%%       Package building/checking facilities
%%     - quality management through R check and tracker _/
%%     - other stuff: forums  _/

%% (5) How to forge packages on R-Forge (A day in a life of a new R
%%     package developer).
%%     - registering as a user  _/
%%     - registering a project (automatically creates SVN, a
%%       mailinglist -> managable through web interface -> click to send
%%       commit statements to this list).  _/
%%     - the svn base tree (picture of an axample tree with more
%%       packages) TODO


\section*{Introduction}

%% from useR abstract 
%% <useR>
%% Why should software developers use source code management tools?

A key factor in open source software (OSS) development is the rapid creation
of solutions within an open, collaborative environment. The open
source model had its major breakthrough with the increasing
usage of the internet. Online communities successfully combined
not only their programming effort but also their knowledge, work
and even their social life. OSS is by definition free and
the source code is available
claimed to be  because of the
``release early, release often'' paradigm introduced by Linus
Torvalds.


\textbf{TODO: } I don't know what to do with the following paragraph
%% FIXME:
Much early work on OSS development was aimed at raising awareness  OSS
work produced by a community of developers ``The Cathedral and the
Bazaar'' by \cite{forge:Raymond:1999}.
Advantages:
decentralized self-organizing process, rapid code evolution, massive
peer code review, rapid release of prototype code
Developers of larger projects like the language \R{} use source code
management (SCM) facilities to follow this paradigm.
The \R{} Development Core Team have been using
development tools like Subversion \citep[SVN,
see][]{forge:Pilato+Collins-Sussman+Fitzpatrick:2004} or  
Concurrent Versions System \citep[CVS, see][]{forge:Cederqvist:2006}
for managing their source code for a decade now.  
A central repository is hosted by ETH Z\"urich mainly for
managing the development of the base \R{} system. The
\R{}-project is now ready to provide similar infrastructure
for the entire \R{} community.
R-Forge (\url{http://R-Forge.R-project.org}) provides a set of tools
for source code management and various web-based
features. It aims to provide a platform for collaborative development of
\R{} packages, \R{}-related software or further projects. But it does not
matter if individuals work on their own or a number of developers
collaborate in a larger project, on R-Forge all work is organized in
the same entity namely a ``Project''.

R-Forge is closely related to the most famous of such platforms---the
world's largest OSS development website---namely
\url{http://SourceForge.net}. In addition to this platform it provides
tools especially for the \R{} community.

%% Article Outline
The present article is organized as follows. First, we present the core
features that R-Forge is offering to the \R{} community. Second, we
give a hands on tutorial on how package developers can get started with 
\R{}-Forge. In particular we show how people can
register a project, use \R{}-Forge's source code management facilities, provide their
packages with \R{}-Forge, host a project specific website, and
finally how to submit a package to CRAN (\url{http://CRAN.R-project.org}).
Eventually we summarize recent developments and give a brief outlook
to future work.

%% What is R-Forge offering to you?
%% Alternative title: Core Features of R-Forge
\section{R-Forge}
%% what is R-Forge?
R-Forge offers a central platform for the development of \R{} packages, \R{}
related software and further projects. 

R-Forge is based on the OSS GForge \citep{forge:copeland_et_al:2006} which is a
fork of the 2.61 SourceForge code maintained by one of the authors of
the original SourceForge code Tim Perdue. GForge has been modified to
provide additional features for the \R{} community namely a
CRAN-style repository for hosting development releases of \R{}
packages as well as a quality management system similar to that of
CRAN.
%%This and the especially of interest for the \R{} community
%% brief overview of features possibly most interesting for readers
Packages hosted on \R{}-Forge are built daily
for various operating systems, namely Linux, Mac OSX and Windows. These
package builds are downloadable from the website of the project on
\R{}-Forge or are installable directly in \R{} via
\code{install.packages("MyPackage",
  repos="http://R-Forge.R-project.org")}.

%%Tabs can be distinguished in main
%%tabs (currently Home, My Page and Project Tree) and project specific
%%tabs which are only visible when you access a project.

%% Additional Information to GForge. Since the codebase of 
%% SourceForge has not been released over a certain time the GForge
%% project was formed. 

%% Why should software developers use source code management tools?
%% FIXME: How to formulate better
%% Version control - probably the most important feature

%%Probably the most important features correspond to 3 project tabs are
%%``Summary'' which offers an overview of the whole project, ``SCM''
%%describing how people can access the source code repository and ``R
%%Packages'' containing a list of packages available in this Project.

%% eventually include history of the development of R-Forge

On R-Forge developers organize their work
in so-called projects. Every project has various tools and web-based features
for software development, communcation between developers themselves
as well as their
user base and other services enabled. All features mentioned in the
following sections are accessible via so called 
``tabs''. E.g., user accounts can be managed in the ``My page'' tab or
a list of available projects can be displayed using the ``Project Tree'' tab. 

Since starting the platform in early 2007 more
and more interested users registered their projects on R-Forge. Now
after a year being in a development and testing stage nearly
200 Projects and around 450  
users are registered on R-Forge. This and the steadily growing list of
feature requests show that there is a high demand for centralized source code
management tools and for releasing prototype code frequently among the
\R{} community.

In the next section we explain the features which are important for
collaborative development and for releasing prototype or development
packages on R-Forge.
%% TABS and projects

\subsection{Source Code Management}

When carrying out a software projects source files change over time,
new files get created and old files deleted. Moreover several authors
work on several computers on the same and/or different files. Often it
is sufficient to have a shared storage for the source code but if the
project reaches a certain size it becomes a tedious task to overlook
every file change. Especially if files get overwritten or deleted it
is often not easy to restore them from a backup. Source
code management tools like SVN help in organizing the process of
creating a software. A central repository ensures that the developer
has always access to the current version of the project source
code. Any of the authorized collaborators can ``checkout'' or
``update'' the project
file structure, make the necessary changes or additions, delete
files from the project and finally ``commit'' his changes or addions
to the repository. More than
that, SVN keeps track of the complete history of the project file
structure. At any point in the development stage it is possible to go
back to any stage in the history as well as to inspect and restore old
files. This is called version control as every stage automatically is
assigned a unique version number which increases over time. 
All of this enables a group of developers to work on a given project
simultaneously without losing time thinking of backups or thinking of
details like throwing away obsolete code as there is always the possibility to 
go back to a specific version of the whole software repository.

%% Summarize the SCM feature
On R-Forge such a version controlled repository is automatically
created for each project. The project members just have to install
the client of their choice (e.g., Tortoise SVN on Windows or svnX on
Mac OSX), check out the repository and can start working using all the
features provided by SVN. In addition to the inherent backup of every
version mentioned above a backup of the whole repository and thus of
the whole history of the project is made daily. 

%This backup is necessary if
%the repository itself gets corrupted. 

%% except two predefined directories (\code{pkg} and \code{www})

%% Rights management on R-Forge
Typically, all users have read access and some of them (the
developers) have write access to the data associated with
a certain project. On R-Forge a rights management system assures that
registered users can be granted one of several roles. 
%% Explaining rights management
E.g., the \textit{Administrator} has all rights including the right to
add new users to the project or release packages directly to CRAN. He
or she is usually the package 
maintainer, the project leader or has registered the project.
Other members of a project typically have either the role \textit{Senior 
Developer} or \textit{Junior Developers}. These two roles are
permitted to commit to the project 
SVN repository and examine the log files in the ``R Packages'' tab.
When we speak of developers in subsequent sections we refer to project
members having the rights of at least of a \textit{Junior Developer}.


\subsection{Release and Quality Management of R Packages}
\label{sec:release_and_quality_management}
%% Quality Management and bug/support/feature tracker
When releasing early version of a software project they are typically
prototypes and therefore are 
not completely bug free. R-Forge can assist the developer with two
tools to improve the quality of their products. First, it offers a
quality management system similar to 
that of CRAN. Packages on R-Forge are checked in a
standardized way based on \code{R CMD check} on different
platforms. The resulting log files can be examined by the project
developers so that they can improve the package to pass all the tests
on R-Forge.

Second, bug tracking systems allow users  to notify
the authors of the package about problems they encountered. 
In the spirit of OSS---given
enough eyeballs, all bugs are shallow
\citep{forge:Raymond:1999}---peer code review leads to an 
overall improvent of the quality of software projects.

\subsection{Additional Features}

Additionally, R-Forge provides other tools especially for larger
projects to coordinate the work between project members and to
communicate with their user base.

\begin{itemize}
\item Project websites are a way for developers to present their work
  on a subdomain of R-Forge (E.g.,
  \url{http://foo.R-forge.R-project.org}). It is also possible to 
  have a link on the project summary page to another website. 
%% checked out hourly from ``www'' directory of the
%%project's repository.
\item Mailing lists: By default one is automatically created when setting up a
  project. Additional mailing lists can be set up as well. 
\item Projects can be categorized into different
  topics (e.g.: biostatistics, finance, regression analysis,
  \ldots). This enables other people to quickly find what they are
  looking for. People can browse the categories in the so-called
  ``Project Tree'' tab. In the default setting the project tree lists
  alphabetically all projects including a short description similar to CRAN.
\item Forums can be set up separately by the project
  administrators.%% are places to discuss certain topics 
\item News can be put on the project summary page as well as on the
  front page. The latter needs approval by one of the R-Forge
  administrators. It is also possible to download them as RSS feeds.
\end{itemize}

\section{How to get started}
%%A small guide through R-Forge---TODO: detailed description 
In this section we show how new users can get started with
R-Forge. For a more detailed guide to R-Forge and additional
information we refer to the user's manual \citep{forge:theussl:2008}.

After accessing the URL \url{http://R-Forge.R-project.org} the main
page is presented to the user. Here one can
\begin{itemize}
\item login,
\item register a user or a project,
\item download the documentation,
\item examine the latest news (also available as RSS feed),
\item enter a specific project website either by search for available
  projects using the search field on the top middle of the page, going
  through an alphabetically listing seen in the
  ``Project Tree'' tab or by clicking on one of the projects listed on
  the right of the front page,
\item or examine R-Forge specific statistics (how many projects/users
  are currently registered on R-Forge, project activity percentages,
  \ldots{}).
\end{itemize}

Another tab on the front page named ``My Page'' brings the user to
their personal page. Among other things one can configure their
account. 

But the first thing upcoming developers have to do is to register as a
new user on R-Forge.

\subsection{Registering as a New User}

Before starting a new project developers have to register themselves
on R-Forge. A link on the main web site called ``New Account'' on the
top right of the page leads to the corresponding registration form.

After submitting the completed form an email is sent to the given mail
address containing a link for activating the account. After the
new account is activated the user is allowed to join an existing project or
create a new one.
The latter is of interest if an
existing package is to be migrated to R-Forge. Therefore we explain
the registration of a project in more details. For joining an existing
project we refer to the user's manual.

\subsection{Registering a Project}

There are two possibilities to register a project. Either clicking on
``Register Your Project'' on the main page  or by going to the ``My
Page'' tab and clicking on ``Register Project''. A form appears which
has to be filled out in order to finish the registration process. Note
that only the ``Project Public Description'' will get visible on R-Forge, whereas
the ``Project Purpose And Summarization'' only is an information for
us---the R-Forge administrators. The ``Project Unix Name'' is the name which
uniquely defines the project (e.g., the name of the
package). Restrictions according to the UNIX file system convention
force Unix names to be in lower case (and will
be converted to it automatically in case they are typed in upper
case).

After submitting the completed form and the approval by the R-Forge
administrators a confirmation email is sent to the user. He or she becomes
the project administrator of the registered project. The SVN
repository as well as the default mailing list are available within
an hour after approval.

%%The project administrator has several
%%possibilities to manage the project. 

After the project has been created it is immediately present in
two web areas on R-Forge. The first one is standardized and the same for every
project. On the ``Summary'' page of the project (typically the initial page after entering a
project) it is possible to

\begin{itemize}
\item examine the details about the project including a short
  description and who are the administrators or developers respectively,
\item follow the download link leading directly to the available packages
  of the project (``R Packages tab''),
\item follow a link leading to the (personal) project homepage,

\item you'll find the latest news announcements (if made any),
\item go to other tabs of activated features of the project like
  ``Forums'', ``Tracker'',
   ``Lists'', ``R Packages'', \ldots{}.
\end{itemize}

The other web area is a personalized one. It is accessible
through e.g., \url{http://foo.R-Forge.R-project.org} and is managed via
the \code{www} directory in the project's SVN repository. Plain html
is supported and the homepage gets updated every hour.

In the next section we show how a package is migrated to R-Forge and
how it can be released to CRAN.

\subsection{SCM and R Packages}

In the ``SCM'' tab of the project it is explained how the SVN
repository can be checked out to your machine. This can be done either
anonymous without write permission or as a developer with full access
to the repository. 

To make make use of the package building and
checking feature introduced in
Section~\ref{sec:release_and_quality_management} the package source
code has to be put into the \code{pkg} directory or, alternatively, a
subdirectory of \code{pkg}. It is possible to have more than one package
in a project. Then the corresponding sources need to be in different
subdirectories of \code{pkg}. After commiting the package sources the
builds will be available within a day in the ``R Packages''
tab. Furthermore, logs of \code{R CMD check} from different platforms
can be examined by the developers.

To migrate an existing repository including the complete history to
R-Forge developers may contact \email{R-Forge@R-project.org}.

The initial created mailing list called
\email{foo-commits@lists.R-Forge.R-project.org} can be used to
circulate commit message of the SVN repository among the
developers. This feature has to be enabled separately in the ``Admin''
section of the ``SCM'' tab. 

To release a package to CRAN use the
``Submit this package to CRAN'' button visible only for the project
administrators in the ``R Packages'' tab. After confirmation that a
message will be sent to \email{CRAN@R-project.org} the package is
provided is automatically copied to
\url{ftp://cran.r-project.org/incoming}.

\section*{Recent and future developments}
In this section we summarize the developments which took place between
the first release basically consisting of the features of GForge and
the current release Version. Then we briefly give an outlook to future
developments.

Recently added features and major changes include:
\begin{itemize}
\item Enhanced structure in the SVN repository allows multiple
  packages in one project. Each package is a subdirectory of the
  \code{/pkg} directory.
\item An additional check box on the news submit page helps to
  separate front page news from project-only news. Nevertheless,
  whenever a news item is declared to be put on the front page,
  R-Forge administrators need to approve them (default project-only
  submission).
\item A new button in the SCM Admin tab allows for delivery of SVN
  commit messages (default: off).  This
  list is used for circulating SVN commit messages to registered
  users (has to be turned on separately).
\item In newly registered projects by default only the SCM, the R
  packages tab and mailing lists are turned on. Forums, News and
  Tracker can be turned on separately using ``Edit Public Info'' in
  the Admin tab of the project. This is aimed to help new users to get
  in touch with R-Forge and to let users decide which features they
  want to use.
\item We moved to a platform independent package building and quality
  management code combined in one \R{} package.
\end{itemize}

Currently on the wishlist and under development are
\begin{itemize}
\item a Wiki,
\item task management facilitis,
\item a reorganized tracker more compatible to \R{}. E.g., \R{}
  version used which generated the bug.
\end{itemize}

\section{Acknowledgements}

Setting up this project would not have been possible without Douglas
Bates and the University of Wisconsin as they provided us with a
server for hosting this platform. Furthermore, 
the authors would like to thank the Computer Center 
of the Wirtschaftsuniversit\"at Wien for
their support and for providing us with additional hardware as well as a
professional server infrastructure.

\bibliography{R-Forge}