\documentclass[a4paper]{article}

\usepackage{amsmath}
\usepackage{amsfonts}
\usepackage[utf8]{inputenc}
\usepackage{url}
%
%\newcommand{\strong}[1]{{\normalfont\fontseries{b}\selectfont #1}}
\newcommand{\class}[1]{\mbox{\textsf{#1}}}
\newcommand{\func}[1]{\mbox{\texttt{#1()}}}
\newcommand{\code}[1]{\mbox{\texttt{#1}}}
\newcommand{\pkg}[1]{\strong{#1}}
\newcommand{\samp}[1]{`\mbox{\texttt{#1}}'}
\newcommand{\proglang}[1]{\textsf{#1}}
\newcommand{\set}[1]{\mathcal{#1}}

\sloppy

\author{Supervisor:\\\\Kurt Hornik\\\\Mentoring Assistant:\\\\ 
        Stefan Theussl}
\title{Data Analysis of Collaboration Networks on R-Forge Using Google's MapReduce}

\begin{document}

\maketitle

%% an abstract and keywords
\begin{abstract} Open Source Software (OSS) development is a classic example
  of collaborative social networks. R-Forge
  (\url{http://R-Forge.R-project.org}) aims to provide a platform for
  collaborative development of \proglang{R} packages,
  \proglang{R}-related software or further projects. On this platform
  developers are connected through shared projects. Every activity
  generated by these people is collected in a central database which
  offers a huge amount of empirical data for investigation. 

  To process these large data sets Google provides a programming
  model and an associated implementation called MapReduce. Programs
  are automatically parallelized and executed on a large cluster of
  workstations. Resources of a distributed system are utilized using a
  \textit{map} function to process a single item and a \textit{reduce}
  function which 
  collects the distributed processed data and combines them to a
  single outcome. 

  The aim of this thesis is to investigate the capabilities of the
  MapReduce implementation and to utilize this framework using
  \proglang{R}. Furthermore, the analysis of the scientific
  collaboration network on R-Forge is a central part of this
  thesis. 

  Potential candidates should have a good command of statistics and
  statistical software as well as a good knowledge in one of the
  programming languages \proglang{C} or \proglang{Java}.
\end{abstract}
%%\Keywords{OSS, MapReduce, social network analysis, \proglang{R}}



%\bibliographystyle{abbrvnat}
%\bibliography{}
%
\end{document}

