\documentclass[a4paper]{article}

\usepackage{amsmath}
\usepackage{amsfonts}
\usepackage[utf8]{inputenc}
\usepackage{url}
%
%\newcommand{\strong}[1]{{\normalfont\fontseries{b}\selectfont #1}}
\newcommand{\class}[1]{\mbox{\textsf{#1}}}
\newcommand{\func}[1]{\mbox{\texttt{#1()}}}
\newcommand{\code}[1]{\mbox{\texttt{#1}}}
\newcommand{\pkg}[1]{\strong{#1}}
\newcommand{\samp}[1]{`\mbox{\texttt{#1}}'}
\newcommand{\proglang}[1]{\textsf{#1}}
\newcommand{\set}[1]{\mathcal{#1}}

\sloppy

\author{Supervisor:\\\\Kurt Hornik\\\\Mentoring Assistant:\\\\ 
        Stefan Theussl}
\title{Applying Parallel Programming Models in the Analysis of
  Collaboration Networks} 

\pagestyle{empty}

\begin{document}

\maketitle

%% an abstract and keywords
\begin{abstract} Open Source Software developing communities are
  a prime example for collaborative social networks. R-Forge
  (\url{http://R-Forge.R-project.org}) aims to provide a platform for
  collaborative development of \proglang{R} packages,
  \proglang{R}-related software and further projects. On this platform
  developers collaborate in so-called projects, i.e., each developer
  can be a member of one or several projects and hence is
  connected to a certain community. What's more, every activity
  of registered users is collected in a central database, meaning that
  a huge amount of data is ready for investigation.  

  To efficiently process large data sets collected with the help of
  their search engine, Google uses a programming model and an
  associated implementation called \textbf{MapReduce}. In principle, functions
  are applied in parallel on parts of the data automatically assigned
  to entities of clusters of workstations by a managing instance. More
  specifically, 
  resources of distributed systems are utilized using a
  \textit{map} function to process a single item and a \textit{reduce}
  function which appropriately collects the distributed processed data
  in order to generate a single outcome.

  The aim of this thesis is to investigate the capabilities of the
  MapReduce implementation and to utilize this framework using
  \proglang{R}. A central application of this thesis is the analysis
  of the scientific collaboration network and the software developing
  community on R-Forge.

  Potential candidates should have a good command of statistics and
  the statistical software \proglang{R} as well as a basic knowledge in
  one of the programming languages \proglang{C} or \proglang{Java}.
\end{abstract}
\vspace{1cm}
\centering \textbf{Keywords:} OSS, MapReduce, social network analysis, \proglang{R}

%\bibliographystyle{abbrvnat}
%\bibliography{}
%
\end{document}

